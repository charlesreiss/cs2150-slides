
\begin{frame}{load factors and chaining}
\begin{itemize}
\item \textit{load factor}: $\lambda = \frac{\text{\# elements}}{\text{table size}}$
\item average number of elements per bucket: $\lambda$
\end{itemize}
\end{frame}

\begin{frame}{find performance}
\begin{itemize}
\item average* time for find: 
    \begin{itemize}
    \item unsuccessful: check $\lambda$ items
    \item successful: check $1+\lambda/2$ items (half of list)
    \end{itemize}
    \vspace{.5cm}
\item *assuming we choose \myemph<2>{random keys?}
\end{itemize}
\end{frame}

% FIXME: choosing load factors

\begin{frame}[fragile,label=badBalance]{maybe our keys aren't average}
\begin{itemize}
\item $h(k) = k$
\item size = $10$
\item index = $h(k) \mod 10$
\vspace{.5cm}
\item $\lambda = 0.3$
\item but if we usually lookup existing keys\ldots
\end{itemize}
\begin{tikzpicture}[overlay,remember picture]
\matrix[tight matrix,
    nodes={font=\small},
    column 1/.style={nodes={draw=none,text width=1.5cm,font=\tt}},
    row 1/.style={nodes={font=\bfseries,draw=none,text depth=1ex}},
    anchor=north east
    ] (arr) at ([xshift=-5cm,yshift=-2cm]current page.north east) {
index \& ~\\
    0 \& |[fill=green!15]| ~ \\
    1 \& ~ \\
    2 \& ~ \\
    3 \& ~ \\
    4 \& ~ \\
    5 \& ~ \\
    6 \& ~ \\
    7 \& ~ \\
    8 \& ~ \\
    9 \& ~ \\
};
\tikzset{
    link/.style={draw,rectangle split,rectangle split parts=2},
    point/.style={-Latex,thick},
    expand/.style={thick,dotted},
}
\node[link,anchor=west,fill=green!15] (link-10) at ([xshift=.75cm]arr-2-2.east) {
    key: {\tt 10} \nodepart{two}
    next:
};
\node[link,anchor=north west,fill=green!15] (link-20) at ([yshift=-.5cm]link-10.south west) {
    key: {\tt 20} \nodepart{two}
    next:
};
\node[link,anchor=north west,fill=green!15] (link-30) at ([yshift=-.5cm]link-20.south west) {
    key: {\tt 30} \nodepart{two}
    next:
};
\draw[expand] (arr-2-2.north east) -- (link-10.north west);
\draw[expand] (arr-2-2.south east) -- (link-10.south west);
\foreach \x/\y in {link-10/link-20,link-20/link-30} {
\draw[point] ([xshift=-2ex]\x.two east) -- ++ (.25cm, 0cm) -- ++ (0cm, -.5cm) -| ([xshift=-.35cm]\y.one west) -- (\y.one west);
};
\draw[point] ([xshift=-2ex]link-30.two east) -- ++(0cm,-.5cm) node[below,font=\small\tt] {NULL};
\end{tikzpicture}
\end{frame}

\begin{frame}{why use a linked list?}
\begin{itemize}
\item \myemph{one item/bucket usually}
\begin{itemize}
    \item if not --- we should use a balanced tree
    \item (or change hash functions?)
\end{itemize}
\item when not one, probably two or three
\item linked list --- probably most efficient
\begin{itemize}
    \item typical space overhead: one NULL pointer
    \item typical time overhead: check the one pointer
\end{itemize}
\end{itemize}
\end{frame}

\begin{frame}{linked list alternatives}
\begin{itemize}
\item vector --- way too much extra space
    \begin{itemize}
    \item size, capacity
    \item extra space reserved in array
    \item remember: typically just one element
    \end{itemize}
\item balanced trees
    \begin{itemize}
    \item two pointers
    \item about same comparisons as linked list for size 2, 3
    \end{itemize}
\end{itemize}
\end{frame}

\begin{frame}{find performance revisited}
\begin{itemize}
\item with \myemph{ideal hash function}: $\Theta(\lambda)$ (load factor)
    \begin{itemize}
    \item typically: adjust hashtable size so $\lambda$ remains approximately constant
    \end{itemize}
\item actual worst case: $\Theta(n)$ (I choose all the wrong keys)
\end{itemize}
\end{frame}

\begin{frame}{insert performance}
\lstset{language=C++,style=small}
\begin{itemize}
\item $\Theta(1)$
    \begin{itemize}
    \item assuming we don't care about checking for a duplicate
    \end{itemize}
\item don't care about sorting the linked list
\item insert at head
\end{itemize}
\end{frame}

\begin{frame}{delete performance}
\begin{itemize}
\item need to do a find to get the bucket
\item then linked list removal $\Theta(1)$
\begin{itemize}
    \item (if singly linked list --- track previous while finding)
\end{itemize}
\end{itemize}
% FIXME: picture
\end{frame}
