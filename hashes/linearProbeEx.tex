\begin{frame}{quadratic probing example}
\begin{itemize}
\item $h(k) = 3k + 7$
\item index = $h(k) \mod 10$
\item then check $h(k) + 1 \mod 10$, $h(k) + 2 \mod 10$, etc.
\item insert \myemph<2>{4}, \myemph<3>{27}, \myemph<4>{37}, \myemph<5>{14}, \myemph<6>{21}
\begin{itemize}
\item $h(k) =$ 1\myemph<2>{9}, 8\myemph<3>{8}, 11\myemph<4>{8}, 4\myemph<5>{9}, 7\myemph<6>{0}
\end{itemize}
\end{itemize}
\begin{tikzpicture}[overlay,remember picture]
\matrix[tight matrix,
    column 1/.style={nodes={draw=none,text width=1.5cm,font=\tt}},
    row 1/.style={nodes={font=\bfseries,draw=none,text depth=1ex}},
    anchor=north east
    ] (arr) at ([xshift=-1cm,yshift=-2cm]current page.north east) {
index \& ~\\
    |[fill=yellow!15]| 0 \& |[alt=<4->{fill=green!15}]| \only<4->{\myemph<4>{37}} \\
    1 \& |[alt=<5->{fill=blue!15}]| \only<5->{\myemph<5>{14}} \\
    2 \& |[alt=<6->{fill=yellow!15}]| \only<6->{\myemph<6>{21}} \\
    3 \& ~ \\
    4 \& ~ \\
    5 \& ~ \\
    6 \& ~ \\
    7 \& ~ \\
    |[fill=green!15]| 8 \& |[alt=<3->{fill=green!15}]| \only<3->{\myemph<3>{27}} \\
    |[fill=blue!15]| 9 \& |[alt=<2->{fill=blue!15}]| \only<2->{\myemph<2>{4}} \\
};
\end{tikzpicture}
\end{frame}

\begin{frame}{the clumping}
\begin{itemize}
\item we tend to get ``clumps'' of used buckets
\item reason why linear probing isn't the only way
\end{itemize}
\end{frame}
