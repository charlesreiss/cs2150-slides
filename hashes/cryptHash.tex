\begin{frame}<1-2>[label=cryptHash]{cryptographic hashes}
\begin{itemize}
\item example: SHA-256
\item input: any string of bits
\item output: 256 bits
\vspace{.5cm}
\item have \myemph{security properties} normal hashes don't:
\begin{itemize}
    \item \myemph<2>{\textbf<2>{collision resistence}}
    \item \myemph<3>{\textbf<3>{preimage resistence}}
\end{itemize}
\end{itemize}
\end{frame}

\begin{frame}{collision resistence}
\begin{itemize}
\item security property of a cryptographic hash
\item it's very  hard to find keys $k_1$ and $k_2$ so $h(k_1) = h(k_2)$
\vspace{.5cm}
\item note: why SHA-256's output is so big (256 bits)
    \begin{itemize}
    \item otherwise, just generate lots of hashes\ldots
    \end{itemize}
\item example application: verify download with hash of file contents
\item it's very hard to find two files with the same hash
\begin{itemize}
\item even if you're trying
\end{itemize}
\end{itemize}
\end{frame}

\begin{frame}[fragile,label=colNonRes]{exercise: collision non-resistence}
\begin{lstlisting}
unsigned int hash(const string &s) {
    unsigned int sum = 0;
    for (int i = 0; i < s.size(); ++i) {
        // deliberate use of wraparound on overflow
        sum *= 37;
        sum += s[i];
    }
    return sum;
}
\end{lstlisting}
\begin{itemize}
\item exercise: how to construct two strings with same hash?
\item<2-> one idea: \{60, $x$\} and \{59, $x+37$\} have the same hash
\end{itemize}
\end{frame}

\againframe<3>{cryptHash}

\begin{frame}{preimage resistence}
\begin{itemize}
\item security property of a cryptographic hash
\item if given $V$, very hard to find $k$ so $h(k) = V$
\end{itemize}
\end{frame}
