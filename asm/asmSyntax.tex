\begin{frame}{two syntaxes}
\begin{itemize}
\item there are two ways of writing x86 assembly
\begin{itemize}
\item AT\&T syntax (default on Linux, OS X)
\item Intel syntax (default on Windows)
\end{itemize}
\item different operand order, way of writing addresses, punctuation, etc.
\item we mostly show Intel syntax
\end{itemize}
\end{frame}

\begin{frame}{different directives}
\begin{itemize}
\item non-instruction parts of assembly are called \textit{directives}
\item IBCM example: \texttt{one   dw  1}
\begin{itemize}\item there is no IBCM instruction called ``dw''\end{itemize}
\item these differ \textit{a lot} between assemblers
\item our main assember: \texttt{NASM} 
\item our compiler's assembler: \texttt{GAS}
\end{itemize}
\end{frame}
