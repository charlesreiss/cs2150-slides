\begin{frame}[fragile,label=usingTheStack]{return addresses using a stack}
\begin{tikzpicture}
\node[draw] (maxAsm) {
\lstset{
    language=myasm,
    style=smaller,
    moredelim={**[is][\btHL<all:2>]{@2}{2@}},
    moredelim={**[is][\btHL<all:3>]{@3}{3@}},
    moredelim={**[is][\btHL<all:4>]{@4}{4@}},
    moredelim={**[is][\btHL<all:5>]{@5}{5@}},
    moredelim={**[is][\btHL<all:6>]{@6}{6@}},
    moredelim={**[is][\btHL<all:7>]{@7}{7@}},
    moredelim={**[is][\btHL<all:8>]{@8}{8@}},
    moredelim={**[is][\btHL<all:9>]{@9}{9@}},
}
\begin{lstlisting}
max:    @5...5@
        @6...6@
        @7ret7@

main:   @2...2@
        @3...3@
        @4call max4@
after:  @8...8@
        @9ret9@
\end{lstlisting}
};
\matrix[tight matrix,nodes={text width=6cm,font=\small},right=1.75cm of maxAsm] (stack) {
    |[alias=retMain,visible on=<2->]| return address for \texttt{main} (OS) \\
    |[alias=tempMain,visible on=<3-8>]| temporary storage for main \\
    |[alias=callOther,visible on=<4-8>]|  other things related to call??? \\
    |[alias=retMax,visible on=<5-7>]| return address for \texttt{max} (\texttt{after:}) \\
    |[alias=maxStorage,visible on=<6>]| temporary storage for max \\
};
\begin{scope}[every node/.style={align=left}]
    \begin{visibleenv}<2>
        \node[above=0cm of stack] {stack when \texttt{main} starts:};
        \node[right=0cm of retMain] {$\leftarrow$ RSP};
    \end{visibleenv}
    \begin{visibleenv}<3>
        \node[above=0cm of stack] {stack in the middle of \texttt{main}:};
        \node[right=0cm of tempMain] {$\leftarrow$ RSP};
    \end{visibleenv}
    \begin{visibleenv}<4>
        \node[above=0cm of stack] {stack just before \lstinline|call max|:};
        \node[right=0cm of callOther] {$\leftarrow$ RSP};
    \end{visibleenv}
    \begin{visibleenv}<5>
        \node[above=0cm of stack] {stack just after \lstinline|call max|:};
        \node[right=0cm of retMax] {$\leftarrow$ RSP};
    \end{visibleenv}
    \begin{visibleenv}<6>
        \node[above=0cm of stack] {stack in the middle of \texttt{max}:};
        \node[right=0cm of maxStorage] {$\leftarrow$ RSP};
    \end{visibleenv}
    \begin{visibleenv}<7>
        \node[above=0cm of stack] {stack just before \texttt{max}'s \lstinline|ret|:};
        \node[right=0cm of retMax] {$\leftarrow$ RSP};
    \end{visibleenv}
    \begin{visibleenv}<8>
        \node[above=0cm of stack] {stack just after \texttt{max}'s \lstinline|ret|:};
        \node[right=0cm of callOther] {$\leftarrow$ RSP};
    \end{visibleenv}
    \begin{visibleenv}<9>
        \node[above=0cm of stack] {stack just before \texttt{main}'s \lstinline|ret|:};
        \node[right=0cm of retMain] {$\leftarrow$ RSP};
    \end{visibleenv}
\end{scope}
\begin{visibleenv}<2->
    \draw[very thick,-Latex] ([xshift=-.25cm,yshift=-.5cm]stack.north west) -- ([xshift=-.25cm,yshift=-.5cm]stack.south west)
        node[below,xshift=1cm]{smaller addresses};
\end{visibleenv}
\end{tikzpicture}
\end{frame}

\begin{frame}{function calls use the stack}
\begin{itemize}
\item ``the'' stack
    \begin{itemize}
    \item convention: RSP points to top
    \item grows `down' (towards address $0$)
    \item used by \texttt{pop}, \texttt{push}, \texttt{call}, \texttt{ret}
    \end{itemize}
\item used to implement function calls
\item main reason: support \myemph{recursive calls}
\item where do (place to return/arguments/local variables/etc.) go?
    \begin{itemize}
    \item when in doubt --- use the stack
    \item optimization: sometimes use registers
    \end{itemize}
\end{itemize}
\end{frame}
