\begin{frame}[fragile,label=cmdLineP]{command line parameters }
\lstset{
    language=C++,
    style=small
}
\begin{lstlisting}
int main (int argc, char* argv[]) { ... }
// same as:
int main (int argc, char **argv) { ... }
\end{lstlisting}
\begin{itemize}
\item \texttt{argc} --- number of arguments + 1
\item \texttt{argv} --- array of \textit{pointers} to C-style strings
    \begin{itemize}
    \item \texttt{argv[0]} --- program name
    \item \texttt{argv[1]}, \texttt{argv[2]}, \ldots --- arguments
    \end{itemize}
\vspace{.5cm}
\item what about \lstinline|int main() { ... }|?
    \begin{itemize}
    \item okay, but can't get arguments
    \end{itemize}
\end{itemize}
\end{frame}

\begin{frame}[fragile,label=argvLayout]
\lstset{
    language=C++,
    style=small
}
\begin{tikzpicture}
\tikzset{
    addrLabel/.style={font=\tt,red!70!black,visible on=<2->}
}
\node (code) {
\begin{lstlisting}
int main(int argc, char **argv) {
    ... // argv == 0x10000
}
\end{lstlisting}
};
\matrix[tight matrix,
    column 1/.style={nodes={draw=none,text width=2.7cm,font=\tt}},
    column 2/.style={nodes={text width=2.5cm,font=\tt,minimum height=.6cm}},
    column 3/.style={nodes={draw=none,text width=2.8cm,font=\tt,addrLabel}},
    row 1/.style={nodes={font=\normalfont}},
    label={[font=\bfseries]north:memory},
    below=0cm of code,
] (mem) {
    address \& value\\
    \ldots \& \ldots  \\
    0x10000-7 \& |[minimum height=1cm]| 0x20000 \& argv[0] \\
    0x10008-F \& |[minimum height=1cm]| 0x2003A \& argv[1]\\
    \ldots \& \ldots \\
    0x20000 \& '.' \& argv[0][0]\\
    0x20001 \& '/' \& argv[0][1] \\
    0x20002 \& 'a' \& argv[0][2] \\
    0x20003 \& '.' \& argv[0][3] \\
    0x20004 \& 'o' \& argv[0][4] \\
};
\matrix[tight matrix,
    column 1/.style={nodes={draw=none,text width=2.5cm,font=\tt}},
    column 2/.style={nodes={text width=1.3cm,font=\tt,minimum height=.6cm}},
    column 3/.style={nodes={draw=none,text width=2.5cm,font=\tt,addrLabel}},
    row 1/.style={nodes={font=\normalfont}},
    anchor=north west
    ] (mem2) at (mem.north east){
    address \& value\\
    0x20005 \& 'u' \& argv[0][5] \\
    0x20006 \& 't' \& argv[0][6] \\
    0x20007 \& '\textbackslash0' \& argv[0][7] \\
    \ldots \& \ldots  \\
    0x2003A \& 'a' \& argv[1][0] \\
    0x2003B \& 'r' \& argv[1][1] \\
    0x2003C \& 'g' \& argv[1][2] \\
    0x2003C \& '1' \& argv[1][3] \\
    0x2003D \& '\textbackslash0' \& argv[1][4] \\
    \ldots \& \ldots \\
};
\end{tikzpicture}
\end{frame}

\begin{frame}[fragile,label=cStringVString]{C strings to strings}
\lstset{
    language=C++,
    style=smaller,
    moredelim={**[is][\btHL<all:1>]{@1}{1@}},
    moredelim={**[is][\btHL<all:2>]{@2}{2@}},
    moredelim={**[is][\btHL<all:3>]{@3}{3@}},
    moredelim={**[is][\btHL<all:4>]{@4}{4@}},
}
\begin{itemize}
\item given a \lstinline|char *c_style_string| (like \lstinline|argv[i]|)
\vspace{.5cm}
\item output:
    \begin{itemize}
    \item \lstinline|cout << c_style_string|
    \end{itemize}
\item convert to C++-style string called \texttt{s}:
    \begin{itemize}
    \item \lstinline|string s(c_style_string)| 
    \item \lstinline|string s; s = c_style_string;| 
    \end{itemize}
\end{itemize}
\end{frame}


\begin{frame}[fragile,label=cmdLinePEx]{command line parameters}
\lstset{
    language=C++,
    style=smaller,
    moredelim={**[is][\btHL<all:1>]{@1}{1@}},
    moredelim={**[is][\btHL<all:2>]{@2}{2@}},
    moredelim={**[is][\btHL<all:3>]{@3}{3@}},
    moredelim={**[is][\btHL<all:4>]{@4}{4@}},
}
\begin{lstlisting}
int main (int argc, char* argv[]) {
    // The 0th command line parameter is the program name.
    cout << "This program is called '" << @2argv[0]2@
         << "'" << endl;
    cout << "The following are the command " 
         << "line parameters you specified: " << endl;
    // for loop starts at 1 to avoid printing
    // name of program (again)
    for ( int i = 1; i < argc; i++ ) {
        // we can convert the C-style strings into 
        // C++-style strings, and then print them:
        string s(@3argv[i]3@);
        cout << "\t" << s << endl;
    }
    return 0;
}
\end{lstlisting}
\end{frame}
