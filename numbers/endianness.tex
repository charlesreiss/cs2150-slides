
\begin{frame}{numbering bits}
\begin{eqnarray*}
\text{option 1: $n$-bit number:}&\;&b_{n-1}b_{n-2}b_{n-3}\ldots b_2b_1b_0 \\
    & = & \sum_{i=0}^{n-1} b_i \cdot 2^i \\
\text{option 2: $n$-bit number:}&\;&b_{0}b_{1}b_{2}\ldots b_{n-3}b_{n-2}b_{n-1} \\
    & = & \sum_{i=0}^{n-1} b_i \cdot 2^{n-i-1} \\
\end{eqnarray*}
\begin{itemize}
\item<2-> two viable ways to number bits
\item<3-> does it matter which I use?
    \begin{itemize}
    \item do I have a way to ask for bit $i$?
    \end{itemize}
\end{itemize}
\end{frame}

\begin{frame}{numbering bytes}
\begin{eqnarray*}
\text{option 1: 4-byte number:}&\;&B_3B_2B_1B_0 \\
    & = & \sum_{i=0}^{3} B_i \cdot 256^i \\
\text{option 2: 4-byte number:}&\;&B_0B_1B_2B_3 \\
    & = & \sum_{i=0}^{3} b_i \cdot 256^{3-i} \\
\end{eqnarray*}
\begin{itemize}
\item<2-> two viable ways to number bytes
\item<3-> does it matter which I use?
    \begin{itemize}
    \item in memory, yes --- each byte needs an address (number)
    \end{itemize}
\end{itemize}
\end{frame}

\begin{frame}[fragile,label=memoryEnd]{memory}
\begin{tikzpicture}
\matrix[tight matrix,
    column 1/.style={nodes={draw=none,text width=1cm}},
    column 2/.style={nodes={text width=2cm,font=\tt}},
    row 1/.style={nodes={font=\normalfont}},
    label={[font=\bfseries,align=center]north:memory \\ (as 64-bit values)}
] (mem64) {
    addr. \& value (64-bit) \\
    0 \& 123999 \\
    8 \& 323232 \\
    16 \& 434093 \\
    \ldots \& \ldots \\
};
    \node[below=1cm of mem64] {
        $\begin{aligned}
            123999 &=& 1\cdot 256^2 \\
                   &+& 228\cdot 256^1 \\
                   &+& 95 \cdot 256^0 \\
        \end{aligned}$
    };
\begin{visibleenv}<2>
\matrix[tight matrix,
    column 1/.style={nodes={draw=none,text width=1cm}},
    column 2/.style={nodes={text width=1cm,font=\tt}},
    row 1/.style={nodes={font=\normalfont,draw=none}},
    label={[font=\bfseries,align=center]north:if \myemph{little endian} \\ (as 8-bit values)},
    anchor=north west,
] (mem8) at ([xshift=1.25cm]mem64.north east) {
    addr. \& value (8-bit) \\
    0 \& 95 \\
    1 \& 228 \\
    2 \& 1 \\
    3 \& 0 \\
    4 \& 0 \\
    5 \& 0 \\
    6 \& 0 \\
    7 \& 0 \\
    8 \& 160 \\
    9 \& 238 \\
    10 \& 4 \\
    11 \& \ldots \\
    \ldots \& \ldots \\
};
\tikzset{
    hiBox/.style={draw=red,inner sep=0.5mm,ultra thick}
}
\node[hiBox,fit=(mem64-2-2)] {};
\node[hiBox,fit=(mem8-2-2) (mem8-9-2)] {};
\draw[red,ultra thick] (mem64-2-2.north east) -- (mem8-2-2.north west);
\draw[red,ultra thick] (mem64-2-2.south east) -- (mem8-9-2.south west);
\matrix[tight matrix,
    column 1/.style={nodes={draw=none,text width=1cm}},
    column 2/.style={nodes={text width=1cm,font=\tt}},
    row 1/.style={nodes={font=\normalfont,draw=none}},
    label={[font=\bfseries,align=center]north:if \myemph{big endian} \\ (as 8-bit values)},
    anchor=north east,
] (mem8B) at ([xshift=-1.25cm]mem64.north west) {
    addr. \& value (8-bit) \\
    0 \& 0 \\
    1 \& 0 \\
    2 \& 0 \\
    3 \& 0 \\
    4 \& 0 \\
    5 \& 1 \\
    6 \& 228 \\
    7 \& 95 \\
    8 \& 0 \\
    9 \& 0 \\
    10 \& 0 \\
    11 \& \ldots \\
    \ldots \& \ldots \\
};
\tikzset{
    hiBox/.style={draw=red,inner sep=0.5mm,ultra thick}
}
\node[hiBox,fit=(mem64-2-2)] {};
\node[hiBox,fit=(mem8-2-2) (mem8-9-2)] {};
\draw[red,ultra thick] (mem64-2-2.north east) -- (mem8-2-2.north west);
\draw[red,ultra thick] (mem64-2-2.south east) -- (mem8-9-2.south west);
\node[hiBox,fit=(mem8B-2-2) (mem8B-9-2)] {};
\draw[red,ultra thick] (mem64-2-2.north west) -- (mem8B-2-2.north east);
\draw[red,ultra thick] (mem64-2-2.south west) -- (mem8B-9-2.south east);
\end{visibleenv}
\end{tikzpicture}
\end{frame}

\begin{frame}[fragile,label=findEndian]{finding endianness in C++}
\lstset{
    language=C++,
    style=smaller,
     moredelim={**[is][\btHL<all:1>]{@1}{1@}},
     moredelim={**[is][\btHL<all:2-3>]{@2}{2@}},
     moredelim={**[is][\btHL<all:4>]{@3}{3@}},
     moredelim={**[is][\btHL<all:5>]{@4}{4@}},
     moredelim={**[is][\btHL<all:6>]{@5}{5@}},
     moredelim={**[is][\btHL<all:7>]{@6}{6@}},
}
\vspace{-.6cm}
\begin{lstlisting}
#include <iostream>
using std::cout; using std::hex; using std::endl;
int main() {
    unsigned long value = @60x0123456789ABCDEF6@;
    cout << hex << value << endl;
    unsigned char *ptr = @2(unsigned char*) &value2@;
    for (int i = 0; i < sizeof(unsigned long); ++i) {
        cout << (int) @3ptr[i]3@ << " ";
    }
    ...
}
\end{lstlisting}
\hrule
\lstset{
    language={},
    style=smaller,
     moredelim={**[is][\btHL<all:1>]{@1}{1@}},
     moredelim={**[is][\btHL<all:2-3>]{@2}{2@}},
     moredelim={**[is][\btHL<all:4>]{@3}{3@}},
     moredelim={**[is][\btHL<all:5>]{@4}{4@}},
     moredelim={**[is][\btHL<all:6>]{@5}{5@}},
     moredelim={**[is][\btHL<all:7>]{@6}{6@}},
}
little endian (e.g. lab machine):
\vspace{-.15cm}
\begin{lstlisting}
@6123456789abcdef6@
@4ef4@ cd ab 89 67 45 23 1
\end{lstlisting}
\hrule
big endian:
\begin{lstlisting}
@6123456789abcdef6@
@515@ 23 45 67 89 ab cd ef
\end{lstlisting}
\begin{tikzpicture}[overlay,remember picture]
\coordinate (place) at ([yshift=4cm]current page.south);
\tikzset{
    box/.style={draw=red,fill=white,ultra thick,align=left,at=(place),inner sep=2mm},
}
\begin{visibleenv}<2>
\node[box] {
    get pointer to \myemph{byte} with \\ lowest address in \texttt{value}
}; 
\end{visibleenv}
\begin{visibleenv}<3>
\node[box] {
    unless you do something like this \\
    won't see endianness
}; 
\end{visibleenv}
\begin{visibleenv}<4>
\node[box] {
    use pointer to get \texttt{i}th byte of value \\
    (cast to int to output as number, not character)
};
\end{visibleenv}
\begin{visibleenv}<5>
\node[box] {
    little endian: byte 0 is \myemph{least significant} \\
    (affects overall value the least)
};
\end{visibleenv}
\begin{visibleenv}<6>
\node[box] {
    big endian: byte 0 is \myemph{most significant} \\
    (affects overall value the most)
};
\end{visibleenv}
\begin{visibleenv}<7>
\node[box] {
    but we don't write numbers in a different order \\
    based on which end we call ``part 0''
};
\end{visibleenv}
\end{tikzpicture}
\end{frame}

\begin{frame}{little versus big endian}
\begin{itemize}
\item little endian --- least significant part has lowest address
\begin{itemize}
\item i.e. index 0 is the one's place
\end{itemize}
\item big endian --- most significant part has the lowest address
\begin{itemize}
\item i.e. index $n-1$ is the one's place
\end{itemize}
\vspace{.5cm}
\end{itemize}
\end{frame}

\begin{frame}{endianness in the real world}
\begin{itemize}
\item today and this course: little endian is dominant
    \begin{itemize}
    \item e.g. x86, \textit{typically} ARM
    \end{itemize}
\item historically: big endian was dominant
    \begin{itemize}
    \item e.g. \textit{typically} SPARC, POWER, Alpha, MIPS, \ldots
    \item still commonly used for networking because of this
    \end{itemize}
\item many architectures have switchable endianness
    \begin{itemize}
    \item e.g. ARM, SPARC, POWER, MIPS
    \item usually, OS chooses one endianness
    \end{itemize}
\end{itemize}
\end{frame}

\begin{frame}{middle endian}
\begin{itemize}
\item sometimes not just big/little endian
\item e.g. number bytes most to least significant as \\
    5, 6, 7, 8, 1, 2, 3, 4
\item e.g. doubles on little-endian ARM
\item generally some sort of historical accident
    \begin{itemize}
    \item e.g. ARM floating point designed for big endian?
    \end{itemize}
\end{itemize}
\end{frame}

\begin{frame}{endianness is about addresses}
\begin{itemize}
\item endianness is about numbering, \\
    not (necessairily) placement on the page
\item but, probably assume English order (left to right, etc.) if not otherwise specified
\end{itemize}
\begin{tikzpicture}
\matrix[tight matrix,
    column 1/.style={nodes={draw=none,text width=1cm}},
    column 2/.style={nodes={text width=1cm,font=\tt}},
    row 1/.style={nodes={font=\normalfont,draw=none}},
    anchor=north west,
] (mem8A) {
    addr. \& value \\
    0 \& 95 \\
    1 \& 228 \\
    2 \& 1 \\
    3 \& 0 \\
    4 \& 0 \\
    5 \& 0 \\
    6 \& 0 \\
    7 \& 0 \\
    8 \& 160 \\
    9 \& 238 \\
    10 \& 4 \\
    11 \& \ldots \\
    \ldots \& \ldots \\
};
\matrix[tight matrix,
    column 1/.style={nodes={draw=none,text width=1cm}},
    column 2/.style={nodes={text width=1cm,font=\tt}},
    row 1/.style={nodes={font=\normalfont,draw=none}},
    anchor=north west,
    right=3cm of mem8A
] (mem8B) {
    addr. \& value \\
    \ldots \& \ldots \\
    11 \& \ldots \\
    10 \& 4 \\
    9 \& 238 \\
    8 \& 160 \\
    7 \& 0 \\
    6 \& 0 \\
    5 \& 0 \\
    4 \& 0 \\
    3 \& 0 \\
    2 \& 1 \\
    1 \& 228 \\
    0 \& 95 \\
};
\node[font=\Huge] at ($(mem8A)!0.5!(mem8B)$) {=};
\end{tikzpicture}
\end{frame}

\begin{frame}{endianness and bit-order}
    \begin{itemize}
    \item we won't talk about \myemph{bit order}
    \item because bits don't have addresses
    \vspace{.5cm}
    \item if I say ``bit $0$'', question: ``numbering from least significant or most significant''?
        \begin{itemize}
        \item nothing about how pointers, etc. work suggests either answer is correct
        \end{itemize}
    \end{itemize}
\end{frame}

\begin{frame}{endianness and writing out bytes}
    \begin{itemize}
        \item \texttt{0x0102} in binary: \texttt{00000001000000010}
            \begin{itemize}
            \item English's order --- most significant first
            \end{itemize}
        \item bytes of \texttt{0x0102} in big endian: \\(byte 0) \texttt{00000001} (byte 1) \texttt{00000010}
        \item bytes of \texttt{0x0102} in little endian: \\(byte 0) \texttt{00000010} (byte 1) \texttt{00000001}
        \vspace{.5cm}
        \item \textit{usually}, we don't change the order we write bits
        \item if writing out bytes, first in reading order is usually lowest address
            \begin{itemize}
            \item (we'll specify if not)
            \end{itemize}
    \end{itemize}
\end{frame}
