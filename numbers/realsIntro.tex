\begin{frame}{some real numbers}
\begin{itemize}
\item $\frac{1}{3}$
\item $-\frac{100}{7}$
\item $\pi$
\item $0.1$
\item $\sqrt{2}$
\item \ldots
\vspace{.5cm}
\item want to represent these: {\small accurately? compactly? efficiently?}
\end{itemize}
\end{frame}

\begin{frame}{fixed point}
\begin{eqnarray*}
\frac{1}{3} &=& 0.101010101\ldots_\text{TWO}  \\\
            &\approx& +0000.1010_\text{TWO} \text{--- represent as {\tt 00000 1010}} \\
\frac{100}{7} &=& 1110.001001001\ldots_\text{TWO}  \\
            &\approx& -1110.0010_\text{TWO} \text{--- represent as {\tt 01110 0010}} \\
\end{eqnarray*}
\begin{itemize}
\item<2-> $x \approx y/2^K$ --- represent with fixed-sized signed integer $y$
\begin{itemize}
    \item this case: $y/2^4$ and $y$ is 9 bits.
\end{itemize}
\end{itemize}
\end{frame}

\begin{frame}{why fixed-point?}
\begin{itemize}
\item $x \approx y/2^K$ ($y$ fixed-sized singed integer)
\item math similar to integer math:
    \begin{itemize}
    \item addition/subtraction --- same
    \item multiplication --- same except divide by $2^K$ 
    \item division --- same except multiply by $2^K$
    \end{itemize}
\item easy to understand what values are represented well
\end{itemize}
\end{frame}

\begin{frame}{why not fixed-point?}
\begin{itemize}
\item pretty small range of numbers for space used
\item hard to choose a $2^K$ that works for lots of applications
\end{itemize}
\end{frame}

\begin{frame}{recall (?): scientific notation}
\begin{eqnarray*}
+\frac{1}{3} &=& +0.33333333\ldots \\
            &\approx& \myemph<3>{+}\myemph<4>{3.33}\cdot \myemph<6>{10}^{\myemph<5>{-1}} \\
-\frac{100}{7} &=& -14.285714\ldots \\
            &\approx& \myemph<3>{-}\myemph<4>{1.42}\cdot \myemph<6>{10}^{\myemph<5>{+1}} \\
\end{eqnarray*}
\begin{itemize}
\item<2-> $\myemph<3>{\pm}\text{\myemph<4>{mantissa}} \cdot \text{\myemph<6>{base}}^\text{\myemph<5>{exponent}}$
\end{itemize}
\end{frame}

\begin{frame}{base-2 scientific notation}
\begin{eqnarray*}
\frac{1}{3} &=& 0.101010101\ldots_\text{TWO}  \\\
            &\approx& 0.1010101010_\text{TWO} =  +1.0101010101_\text{TWO} \cdot 2^-1\\
-\frac{125}{4} &=& -111111.01 \ldots_\text{TWO}  \\\
            &=&-1.1111101_\text{TWO} \cdot 2^2\\
-\frac{100}{7} &=& -1110.01001001\ldots_\text{TWO}  \\\
            &\approx& -1110.010010_\text{TWO} = -1.1100100101_\text{TWO} \cdot 2^3\\
\end{eqnarray*}
\end{frame}


