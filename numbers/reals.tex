\begin{frame}{some real numbers}
\begin{itemize}
\item $\frac{1}{3}$
\item $-\frac{100}{7}$
\item $\pi$
\item $0.1$
\item $\sqrt{2}$
\item \ldots
\vspace{.5cm}
\item want to represent these:
    \begin{itemize}
    \item accurately? compactly? efficiently?
    \end{itemize}
\end{itemize}
\end{frame}

\begin{frame}{fixed point}
\begin{eqnarray*}
\frac{1}{3} &=& 0.101010101\ldots_\text{TWO}  \\\
            &\approx& 0.1010_\text{TWO} \text{--- represent as {\tt 00000 1010}} \\
\end{eqnarray*}
\begin{eqnarray*}
\frac{100}{7} &=& 1110.001001001\ldots_\text{TWO}  \\\
            &\approx& -1110.0010_\text{TWO} \text{--- represent as {\tt 01110 0010}} \\
\end{eqnarray*}
\begin{itemize}
\item<2-> $x \approx y/2^K$ --- $y$ is a fixed-sized integer
\begin{itemize}
    \item this case: $y/2^4$ and $y$ is 9 bits.
\end{itemize}
\item<3-> can also use non-power-of-two base
    \begin{itemize}
    \item e.g. representing dollars+cents as a number of cents
    \end{itemize}
\end{itemize}
\end{frame}

\begin{frame}{why fixed-point?}
\begin{itemize}
\item $x \approx y/2^K$ ($y$ fixed-sized integer)
\item math similar to integer math:
    \begin{itemize}
    \item addition/subtraction --- same
    \item multiplication --- same except divide by $2^{2K}$ at the end
    \item division --- same
    \end{itemize}
\item easy to understand what values are represented well
\end{itemize}
\end{frame}

\begin{frame}{why not fixed-point?}
\begin{itemize}
\item pretty small range of numbers for space used
\item hard to choose a $2^K$ that works for lots of applications
\end{itemize}
\end{frame}

\begin{frame}{recall (?): scientific notation}
\begin{eqnarray*}
+\frac{1}{3} &=& +0.33333333\ldots \\
            &\approx& \myemph<3>{+}\myemph<4>{3.33}\cdot 10^{\myemph<5>{-1}} \\\
\end{eqnarray*}
\begin{eqnarray*}
-\frac{100}{7} &=& -14.285714\ldots \\
            &\approx& \myemph<3>{-}\myemph<4>{1.42}\cdot 10^{\myemph<5>{+1}} \\\
\end{eqnarray*}
\begin{itemize}
\item<2-> $\myemph<3>{\pm}\text{\myemph<3>{mantissa}} \cdot \text{\myemph<5>{base}}^\text{\myemph<4>{exponent}}$
\item<3-> sign --- plus or minus (one bit)
\item<4-> mantissa (``value'' of number)
    \begin{itemize}
    \item fixed precision
    \item $1 \le \text{mantissa} < \text{base}$
    \end{itemize}
\item<5-> exponent
\item<6-> base
    \begin{itemize}
    \item normal scientific notation: 10
    \item typical computers: 2
    \end{itemize}
\end{itemize}
\end{frame}

\begin{frame}{base-2 scientific notation}
\begin{eqnarray*}
\frac{1}{3} &=& 0.101010101\ldots_\text{TWO}  \\\
            &\approx& 0.1010101010_\text{TWO} = +1.0101010_\text{TWO} \cdot 2^-1\\
\end{eqnarray*}
\begin{eqnarray*}
\frac{100}{7} &=& 1110.001001001\ldots_\text{TWO}  \\\
            &\approx& -1110.0010010_\text{TWO} = -1.1100010010_\text{TWO} \cdot 2^3\\
\end{eqnarray*}
\end{frame}

\begin{frame}{IEEE half-precision floating point}
\begin{tikzpicture}
\tikzset{
    >=Latex,
    labelLine/.style={draw,thick,dotted},
    myLabel/.style={align=center,font=\small},
    pm/.style={blue!70!black},
    mant/.style={green!60!black},
    expt/.style={orange!80!black},
}
\begin{scope}[every node/.style={inner sep=0mm}]
\node[pm] (pm) {-};
\node[anchor=base west] (onePt) at (pm.base east) {$1.$};
\node[anchor=base west,mant] (mant) at (onePt.base east) {$1100010010_\text{TWO}$};
\node[anchor=base west] (cdot) at (mant.base east) {\cdot};
\node[anchor=base west] (expt) at (cdot.base east) {$2^{\color{orange!80!black}3}$};
\end{scope}
\begin{visibleenv}<2->
\node[pm,below left=1cm of pm,myLabel] (signLabel) {sign (1 bit) };
\draw[pm,labelLine] (pm) -- (signLabel);
\node[mant,right=.25cm of signLabel,myLabel] (mantLabel) {mantissa ($M=10$ bits)};
\draw[mant,labelLine] (mant) -- (mantLabel);
\node[right=.25cm of mantLabel,myLabel,expt] (exptLabel) {exponent ($E=5$ bits)};
\draw[expt,labelLine] (expt) -- (exptLabel);
\node[below=1cm of mantLabel] {
    \tt {\color{blue!70!black}1} {\color{orange!80!black}10010} {\color{green!60!black}1100010010}
};
\end{visibleenv}
\begin{visibleenv}
\item<2-> \color{blue!70!black} sign bit: {\tt 0} for $+$, {\tt 1} for $-$
\item<3-> \color{orange!80!black} exponent $E$: $E+15=e$ stored as unsigned 
    \begin{itemize}
    \item 15 called the ``bias'' --- $15 = 2^5-1$
    \item this example: $3+15= 18 = 10010_\text{TWO}$
    \end{itemize}
\item<4-> \color{green!60!black} mantissa: stored \myemph{without leading ``1.''}
    \begin{itemize}
    \item ``1.'' always present in base-2 scientific notation
    \end{itemize}
\end{tikzpicture}
\end{frame}

\begin{frame}{IEEE half precision float}
\begin{itemize}
\item 1 sign bit ($1$ for negative)
\item 5 expontent bits
    \begin{itemize}
    \item bias of 15 --- if bits as unsigned are $e$, exponent is $E=e-127$
    \end{itemize}
\item 10 mantissa bits
    \begin{itemize}
    \item leading ``1.'' not stored
    \end{itemize}
\vspace{.5cm}
\item $\text{value} = (1-2\cdot \text{sign}) \cdot (1.\text{mantissa}_\text{TWO}) \cdot 2^{\text{exponent}-15}$
\end{itemize}
\end{frame}

\begin{frame}{other IEEE precisions}
\begin{tabular}{lll}
C++/Java type & --- & \texttt{float} & \texttt{double} \\
sign bits & 1 & 1 & 1 \\
exponent bits & 5 & 8 & 11 \\
exponent bias & 15 ($2^5-1$) & 127 ($2^7-1$) & 1023 ($2^10-1$)\\
mantissa bits & 10 & 23 & 52 \\
total bits & 16 & 32 & 64 \\
\end{tabular}
\end{frame}

% FIXME: single precision examples

% FIXME: IEEE format
% FIXME: bias representation for exponents
% FIXME: implciit 1 on mantissa
% FIXME: 0.1 can't be represented
% FIXME: zero
% FIXME: infinity
% FIXME: denormal
% FIXME: converting to/from IEEE format
% FIXME: catastrophic cancellation
% FIXME: not associative
