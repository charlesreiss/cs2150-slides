\begin{comment}
\begin{frame}{last time}
    \begin{itemize}
        \item implementing stacks
        \item queues
        \item abstract data type --- data + set of operations on it
            \begin{itemize}
            \item can be represented as a class or not
            \item multiple, interchangeable representations
            \end{itemize}
        \item base-$X$ numbers: $ab.ef\rightarrow aX^1+bX^0+eX^{-1}+fX^{-2}$
        \item C constants: {\tt 99} (base 10), {\tt 0143} (base 8), {\tt 0x63} (base 16)
        \item converting to base $X$ (one way)
            \begin{itemize}
            \item divide by X, remainder is last digit
            \item divide by X again, remainder is next to last digit
            \item \ldots
            \end{itemize}
    \end{itemize}
\end{frame}

\begin{frame}[fragile,label=realCpp]{on real numbers in C++}
\lstset{language=C++,style=small}
    \begin{itemize}
        \item \lstinline|x = 009;| --- syntax error
        \item \lstinline|x = 070;| --- base 8, same as \lstinline|x = 56;|
        \item \lstinline|x = 009.9;| --- base 10, same as \lstinline|x = 9.9;|
        \item \lstinline|x = 070.0;| --- base 10, same as \lstinline|x = 70.0;|
    \end{itemize}
\end{frame}

\begin{frame}{last time}
    \begin{itemize}
    \item why bits and not base-10?
    \item endianness
    \item maximum unsigned integers
    \item started representing negative numbers
        \begin{itemize}
        \item key property: same bits as unsigned numbers if in range
        \item top bit becomes ``sign bit'' (1 if negative, 0 otherwise)
        \end{itemize}
    \end{itemize}
\end{frame}
\end{comment}

\begin{frame}{last time}
    \begin{itemize}
    \item representating negative numbers
    \begin{itemize}
    \item common property: top bit is ``sign bit'' --- 1=negative, 0=positive
    \item sign and magnitude: flip sign bit to negate number
    \item 1's complement: flip all bits to negate number
    \item 2's complement: flip all bits and add one to negate number
    \end{itemize}
    \item 2's complement: add/subtraction same bits as unsigned
    \item fixed point --- store $Y$ to represent $Y/2^K$
    \item floating point --- store base 2 scientific notation
    \end{itemize}
\end{frame}
