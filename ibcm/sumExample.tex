\begin{frame}{example: sum}
\begin{itemize}
\item the task:
\vspace{1cm}
\item read in integer $n$ from keyboard
\item compute sum of integers $1$ to $n$ (inclusive)
\item print sum
\item halt
\end{itemize}
\end{frame}

\begin{frame}[fragile,label=sumPsuedo]{sum psuedocode}
\begin{lstlisting}
read n;
i = 1;       // index in the array
s = 0;       // ongoing sum
while (i <= n) {
  s += i;
  i += 1;
}
print s;
\end{lstlisting}
\end{frame}

% FIXME: incomplete
\begin{frame}[fragile,label=sumTranslate]{translating sum}
\lstset{
    style=small,
    language={},
}
\begin{tikzpicture}
\tikzset{
    outBox/.style={draw,thick},
    inBox/.style={draw,thick},
}
\node[inBox] (original) {
\begin{Verbatim}[commandchars=\\\{\}]
read \myemph<2>{n};
\myemph<2>{i} = 1;
\myemph<2>{s} = 0;
while (i <= n) {
  s += i;
  i += 1;
}
print s;
\end{Verbatim}
}
\begin{visibleenv}<2->
\node[outBox,right=1cm of original] (prologue) {
\begin{lstlisting}
    jmp     start
i   dw      0     
s   dw      0     
n   dw      0
\end{lstlisting}
};
\node[explain,right=1cm of prologue] {
allocate memory for each variable \\
set initial values to all 0s \\
\texttt{jmp} to not try to execute variables
};
\end{visibleenv}
\end{tikzpicture}
\end{frame}
