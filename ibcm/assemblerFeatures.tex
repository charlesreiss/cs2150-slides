\begin{frame}{ICBM assembly language}
    \begin{itemize}
    \item don't have an assembler implemented
    \item \ldots but let's see what an assembly language would look like
    \end{itemize}
\end{frame}

\begin{frame}<1-2>[label=assemblyFeatures]{ICBM assembler}
    \begin{itemize}
    \item assembly: {\tt load \myemph<2->{0x100}} \\ $\rightarrow$ opcode={\tt 3}, addr={\tt0x100} \\
        machine code: {\tt 0011 000100000000} \\
    \item assembly: {\tt add \myemph<2->{0x200}} \\ $\rightarrow$ opcode={\tt 5}, addr={\tt 200} \\
        machine code: {\tt 0101 001000000000} \\
    \item assembly: {\tt jmpe \myemph<2->{0x442}} \\ $\rightarrow$ opcode={\tt D}, addr={\tt 442} \\
        machine code: {\tt 1101 010001000010}\\
    \end{itemize}
    \begin{tikzpicture}[overlay,remember picture]
    \begin{visibleenv}<2->
    \node [anchor=south,align=left,draw=red,ultra thick] at ([yshift=.1cm]current page.south) {
    \myemph<2>{work with hard-coded addresses?} \\
    \myemph<3>{how to set initial values?}
    };
    \end{visibleenv}
    \end{tikzpicture}
\end{frame}

\begin{frame}[fragile,label=addrAsName]{labels: addresses as names}
\begin{lstlisting}[language=myasm]
add     100     // addr 0: a += mem[100]
jmpl    3       // addr 1: if a < 0: goto 3
jmp     0       // addr 2: [otherwise] goto 0
nop             // addr 3: do nothing
\end{lstlisting}
\hrule
\begin{visibleenv}<2->
\begin{lstlisting}[language=myasm]
start  add     100     // addr 0: a += mem[100]
       jmpl    end     // addr 1: if a < 0: goto 3
       jmp     start   // addr 2: [otherwise] goto 0
end    nop             // addr 3: do nothing
\end{lstlisting}
\end{visibleenv}
\end{frame}

\begin{frame}[fragile,label=addrAsName2]{labels: addresses as name}
\begin{lstlisting}[language=myasm]
start  add     100     // addr 0: a += mem[100]
       jmpl    end     // addr 1: if a < 0: goto 3
       jmp     start   // addr 2: [otherwise] goto 0
end    nop             // addr 3: do nothing
\end{lstlisting}
    \begin{itemize}
    \item name for a \myemph{memory address}
    \item address of instruction or of data
    \item replaced by address when executable is produced
    \end{itemize}
\end{frame}

\againframe<3>{assemblyFeatures}

\begin{frame}{assembly directives}
    \begin{itemize}
    \item not everything in assembly is instructions
    \item program data, strings, etc.
    \item assemblers have \myemph{directives}
    \item processed by assembler to produce special output
    \vspace{.5cm}
    \item<2-> \textbf{dw} directive (``define word'')
    \item<3-> \texttt{i dw 75}
        \begin{itemize}
            \item place the value \texttt{75} in memory
            \item name the address where it is placed \texttt{i}
        \end{itemize}
    \end{itemize}
\end{frame}

\begin{frame}[fragile,label=dwExample]{example with dw}
\begin{lstlisting}[language=myasm,style=small]
loop    load hundred   // a <- 100
        jmpl end       // if a < 0: goto end
        printH         // print a
        sub one        // a <- a - 1
        jmp loop
end     halt

hundred dw 100
one     dw 1
\end{lstlisting}
\hrule
\begin{lstlisting}[language=C++,style=small]
int a = 100;
while (a >= 0) {
    print a;
    a -= 1;
}
\end{lstlisting}
\end{frame}

\begin{frame}[fragile,label=dwVarExample]{variables with dw}
\begin{lstlisting}[language=myasm,style=small]
        load i
        add j
        store i

        load j
        sub i
        sub i
        store j

i       dw 10
j       dw 20
\end{lstlisting}
\hrule
\begin{lstlisting}[language=C++,style=small]
int i = 10, j = 20;
i += j;
j -= i;
j -= i;
\end{lstlisting}
\end{frame}
