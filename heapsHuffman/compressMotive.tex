\begin{frame}{compression}
\begin{itemize}
\item compression
\vspace{.5cm}
\item 50KB webpage as 5KB download (a lot faster!) 
\item 100MB of machine code as 50MB download?
\item movie of 24 1MB pictures/second into 10MB/minute file?
\item \ldots
\end{itemize}
\end{frame}

\begin{frame}{lossy compression}
\begin{itemize}
\item for audio, pictures, video, \textit{lossy compression} is common
\item intuition: you won't notice if we make the pixel 0.25\% darker
    \begin{itemize}
    \item \ldots and it had ``noise'' from camera sensor, etc. anyways
    \end{itemize}
\item idea: model human perception
\item write down \myemph{most important parts} of audio/image/etc.
    \begin{itemize}
    \item important = noticed by humans
    \end{itemize}
\end{itemize}
\end{frame}

\begin{frame}{lossless compression}
\begin{itemize}
\item lossless compression --- reproduce original file
\item rely on patterns
\vspace{.5cm}
\item example: text file has many more `e's than '!'s
\begin{itemize}
\item \ldots so choose shorter encoding for `e' than `!'
\end{itemize}
\item example: computer-drawn images have lots of white space
\begin{itemize}
\item \ldots so have a way to represent ``a big white rectangle'' (instead of specifying each pixel)
\end{itemize}
\end{itemize}
\end{frame}

\begin{frame}{typical compression results}
\begin{itemize}
\item ratio = original size:final size
\item note: usually a compression ratio/speed tradeoff (not shown)
\item lossless:
    \begin{itemize}
    \item for English text or source code: about 4:1
    \item for CD-quality audio: about 2:1
    \item for photographs: about 2:1
    \item for computer-drawn diagrams: about 5:1 to 20:1
    \end{itemize}
\item lossy: (making a guess at what is ``close enough'' in quality)
    \begin{itemize}
    \item for CD-quality audio: about 4:1
    \item for standard definition TV video+audio: about 1:40
    \end{itemize}
\end{itemize}
\end{frame}
