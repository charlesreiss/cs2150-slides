\usetikzlibrary{graphs,graphdrawing}
\usegdlibrary{trees}

\begin{frame}[fragile,label=treesAsArrays]{trees as arrays}
\lstset{language=C++,style=smaller}
\begin{tikzpicture}
\tikzset{
    >=Latex,
    mybst/.style={binary tree layout,level distance=5mm,sibling distance=4mm,nodes={draw,circle,inner sep=0.5mm,minimum width=.8cm}},
    missing/.style={invisible,text=white},
    pointerA/.style={->,very thick,green},
    pointerB/.style={->,very thick,blue},
    pointerC/.style={->,very thick,violet},
    pointerAR/.style={<-,very thick,green},
    pointerBR/.style={<-,very thick,blue},
    pointerCR/.style={<-,very thick,violet},
}
\begin{scope}[mybst]
\graph {
    [name=a] A[desired at={(0,0)}] -> {
        B -> {D[> alt=<2>{pointerBR},> alt=<3>{pointerB}] -> {H[> alt=<3>{pointerC}], I[> alt=<2>{pointerAR},> alt=<3>{pointerA}]}, E -> {J, Xb[missing,> missing]}},
        C -> {F ->[missing] {Xc[missing], Xd[missing]}, G ->[missing] {Xe[missing], Xf[missing]}}
    }
};
\end{scope}
\matrix[tight matrix,
    nodes={font=\small,text width=.5cm},
    column 1/.style={nodes={text width=1.5cm,draw=none,font=\small\bfseries}},
] (asArray) at (1, -6) {
   node  \& ~ \& A \& B \& C \& D \& E \& F \& G \& H \& I \& J  \& ~  \& ~  \& ~  \& ~  \& ~  \& ~ \\
   index \& 0 \& 1 \& 2 \& 3 \& 4 \& 5 \& 6 \& 7 \& 8 \& 9 \& 10 \& 11 \& 12 \& 13 \& 14 \& 15 \& 16 \\
};
\node[below=0cm of asArray] {
\lstset{language=C++,style=smaller}
\begin{lstlisting}
string theTree[17] = {"", "A", "B", ....}
\end{lstlisting}
};
\coordinate (rules) at (4, 0);
\begin{visibleenv}<2>
\draw[pointerA] (asArray-1-11.north) -- ++(0cm, .5cm) -| ([xshift=.25cm]asArray-1-6);
\draw[pointerB] ([xshift=-.25cm]asArray-1-6.north) -- ++(0cm, .5cm) -| (asArray-1-4);
\end{visibleenv}
\begin{visibleenv}<2->
\node[anchor=north west,alt=<2>{red},inner sep=0mm] (parentRule) at (rules) {
\lstinline|parentIndex = index / 2|
};
\end{visibleenv}

\begin{visibleenv}<3>
\draw[pointerAR] (asArray-1-11.north) -- ++(0cm, .5cm) -| ([xshift=.25cm]asArray-1-6);
\draw[pointerBR] ([xshift=-.25cm]asArray-1-6.north) -- ++(0cm, .5cm) -| (asArray-1-4);
\draw[pointerCR] (asArray-1-10.north) -- ++(0cm, .4cm) -| ([xshift=.35cm]asArray-1-6);
\end{visibleenv}
\begin{visibleenv}<2->
\node[anchor=north west,alt=<2>{red},align=left,inner sep=0mm] at ([yshift=.2cm]parentRule.south west) {
\lstinline|leftChild = index * 2| \\
\lstinline|rightChild = index * 2 + 1| \\
};
\end{visibleenv}
\end{tikzpicture}
\end{frame}

\begin{frame}{why arrays}
\begin{itemize}
\item single array --- less storage/memory allocation
\item represent tree as single vector
\end{itemize}
\end{frame}
