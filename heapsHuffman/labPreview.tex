\begin{frame}{lab preview}
\begin{itemize}
\item pre-lab: compression
\item in-lab: decompression
\item post-lab report
\end{itemize}
\end{frame}

\begin{frame}{pre-lab}
\begin{itemize}
\item write a program to\ldots
\vspace{.5cm}
\item calculate letter frequencies of input
\item use binary heap to build huffman tree
\item output encoding mapping (format specified in lab)
\item output encoded message
\end{itemize}
\end{frame}

\begin{frame}{pre-lab tools}
\begin{itemize}
\item heap code supplied in slides
\item file I/O code provided (\texttt{fileio.cpp})
\begin{itemize}
\item or see \texttt{getWordInTable.cpp} from lab 6
\item or see \url{http://www.cplusplus.com/doc/tutorial/files/}
\item or see \href{http://en.cppreference.com/w/cpp/io/basic_ifstream}{ifstream documentation}
\end{itemize}
\end{itemize}
\end{frame}

\begin{frame}{a note on ASCII}
\begin{itemize}
\item the American standard character codes
\begin{itemize}
\item 7-bit charcters (extra bit left over in bytes)
\item ASCII or superset used to represent English text
\end{itemize}
\item 128 characters (95 printable, 33 non-printable)
\item \href{https://en.wikipedia.org/wiki/ASCII}{Wikipedia article} as table/details
\end{itemize}
\end{frame}

\begin{frame}{ASCII codes}
\begin{itemize}
\item for lab: only worry about ``printable'' ASCII characters
\begin{itemize}
\item byte values \texttt{0x20} to \texttt{0x7e}
\end{itemize}
\item special case: \texttt{0x20} = `space'
\item no other whitespace characters used
\begin{itemize}
\item (output character in table as itself\ldots)
\end{itemize}
\end{itemize}
\end{frame}
