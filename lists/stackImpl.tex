\begin{frame}{stack implmentation choices}
\begin{itemize}
\item need to keep track of multiple items
\item several data structures for doing so\ldots
\vspace{.5cm}
\item \myemph<2>{singly linked lists}
\item doubly linked lists
\item arrays
\item\ldots
\end{itemize}
\end{frame}

\makeatletter
\tikzset{
    prefix node name/.code={%
        \tikzset{name/.code={\edef\tikz@fig@name{#1 ##1}}}%
    },
}
\makeatother

% FIXME: source?
\begin{frame}[fragile,label=llStackOfInts]{linked list stack of ints}
\lstset{language=C++,style=smaller}
\begin{tikzpicture}
    \tikzset{
        codeBox/.style={},
        classBox/.style={very thick,draw,rectangle split,rectangle split parts=2,font=\small,align=left,
            inner sep=0.5mm,text depth=0mm},
        classBoxLabel/.style={font=\tt\small},
        classBoxB/.style={classBox,rectangle split parts=3,align=left},
        pointerLine/.style={draw,ultra thick},
        null/.pic={
            \draw (-2mm,0mm) -- (2mm, 0mm);
            \draw (-1.5mm,-0.75mm) -- (1.5mm, -0.75mm);
            \draw (-1mm,-1.5mm) -- (1mm, -1.5mm);
        },
        >=Latex,
    }
\node[codeBox] (mainCode) {
\begin{lstlisting}
class StackNode {
    ...
    int value;
    StackNode *next;
};

class Stack {
  public:
    Stack();
    ~Stack();
    bool isEmpty() const;
    int top() const;
    void push(int value);
    void pop();

  private:
    StackNode *head;
};
\end{lstlisting}
};
\begin{visibleenv}<2>
\node[codeBox,anchor=north west] (exA-aStackCode) at ([xshift=.5cm]mainCode.north east) {
\begin{lstlisting}
Stack aStack;
\end{lstlisting}
};
\begin{scope}[name prefix=exA-]
\node[below=1cm of aStackCode,draw,thick,classBoxB,label={[classBoxLabel]north:aStack}] (aStack) {
\bfseries Stack \nodepart{two}
    head \hspace{.5cm} \nodepart{three}
Stack()  \\
\textasciitilde{}Stack() \\
isEmpty() \\
    push() \\
    pop()
};
    \path[pointerLine,->] ([xshift=-2ex]aStack.two east) -- ++(1cm,0cm) -- ++(0cm,-3cm) coordinate (forNull);
    \path (forNull) pic {null};
\end{scope}
\end{visibleenv}

\begin{visibleenv}<3>
\node[codeBox,anchor=north west] (exB-aStackCode) at ([xshift=.5cm]mainCode.north east) {
\begin{lstlisting}
Stack aStack;
aStack.push(1);
\end{lstlisting}
};
\begin{scope}[name prefix=exB-]
\node[below=1cm of aStackCode,draw,thick,classBox,label={[classBoxLabel]north:aStack}] (aStack) {
\bfseries Stack \nodepart{two}
head
};
    \node[below=1cm of aStack,xshift=1cm,classBox] (stackNodeA) {
\bfseries StackNode \nodepart{two}
value = 1 \\
next
};
    \path[pointerLine,->] ([xshift=-1ex]aStack.two east) -| ++(.5cm,-.75cm) -| ([xshift=-.5cm]stackNodeA.one west) -- (stackNodeA.one west);
    \path[pointerLine,->] ([yshift=-1.5ex,xshift=-1cm]stackNodeA.two east) -- ++(.5cm,0cm) -- ++(0cm,-1cm)
        coordinate (forNull);
    \path (forNull) pic {null};
\end{scope}
\end{visibleenv}

\begin{visibleenv}<4>
\node[codeBox,anchor=north west] (exC-aStackCode) at ([xshift=.5cm]mainCode.north east) {
\begin{lstlisting}
Stack aStack;
aStack.push(1);
aStack.push(2);
\end{lstlisting}
};
\begin{scope}[name prefix=exC-]
\node[below=0.5cm of aStackCode,draw,thick,classBox,label={[classBoxLabel]north:aStack}] (aStack) {
\bfseries Stack \nodepart{two}
head
};
    \node[below=0.5cm of aStack,xshift=1cm,classBox] (stackNodeA) {
\bfseries StackNode \nodepart{two}
value = 2 \\
next
};
    \node[below=0.5cm of stackNodeA,classBox] (stackNodeB) {
\bfseries StackNode \nodepart{two}
value = 1 \\
next
};
    \path[pointerLine,->] ([xshift=-1ex]aStack.two east) -| ++(.5cm,-.5cm) -| ([xshift=-.5cm]stackNodeA.one west) -- (stackNodeA.one west);
    \path[pointerLine,->] ([xshift=-1cm,yshift=-1ex]stackNodeA.two east) -| ++(.5cm,-.5cm) -| ([xshift=-.5cm]stackNodeB.one west) -- (stackNodeB.one west);
    \path[pointerLine,->] ([yshift=-1ex,xshift=-1cm]stackNodeB.two east) -- ++(.5cm,0cm) -- ++(0cm,-1cm)
        coordinate (forNull);
    \path (forNull) pic {null};
\end{scope}
\end{visibleenv}
\end{tikzpicture}
\end{frame}

\begin{comment}
\begin{frame}[fragile,label=implLLStack]{implementing linked list stack}
\lstset{language=C++,style=small}
\begin{lstlisting}
bool Stack::isEmpty() cosnt {
    return head == NULL;
}

int Stack::top() const {
    // FIXME: throw exception if empty?
    return head->value;
}
\end{lstlisting}
\end{frame}
\end{comment}

\begin{frame}[fragile,label=vectorStack]{vector stack of ints}
\lstset{language=C++,style=smaller}
\begin{lstlisting}
class Stack {
  public:
    Stack();
    ~Stack();
    bool isEmpty() const;
    int top() const;
    void push(int value);
    void pop();

  private:
    vector<int> data;
};
\end{lstlisting}
\begin{itemize}
\item \texttt{data} contains elements of stack
\item last element of \texttt{data} is ``top''
    \begin{itemize}
    \item (lets \texttt{push} be fast)
    \end{itemize}
\end{itemize}
\end{frame}

\begin{frame}[fragile,label=vectorStackImpl]{implementing vector stack}
\lstset{language=C++,style=small}
\begin{lstlisting}
bool Stack::isEmpty() const {
    return data.size() == 0;
}

void Stack::push(int value) {
    data.push_back(value);
}

void Stack::pop() {
    data.pop_back();
}

// ...
\end{lstlisting}
\end{frame}

\begin{frame}[fragile,label=vectorStackImplPop]{implementing pop?}
\lstset{language=C++,style=small}
\begin{lstlisting}
void Stack::pop() {
    ...
}
\end{lstlisting}
What could go here? \\
    \begin{tabular}{l}
        A. \lstinline|data.pop_front();| \\
        B. \lstinline|data.resize(data.size() - 1);| \\
        C. \lstinline|data.reserve(data.size() - 1);| \\
        D. \lstinline|data.erase(data.begin());| \\
        E. \lstinline|data.pop_back();| \\
    \end{tabular}
\begin{visibleenv}<2>
\iftoggle{heldback}{}{
\\
\myemph{B or E}
}
\end{visibleenv}
\end{frame}

\begin{frame}[fragile,label=vectorStackImplTop]{implementing top?}
\lstset{language=C++,style=small}
\begin{lstlisting}
int Stack::top() {
    return ...
}
\end{lstlisting}
What could go here? \\
    \begin{tabular}{l}
        A. \lstinline|data.back();| \\
        B. \lstinline|data.at(data.size());| \\
        C. \lstinline|data.at(data.size() - 1);|\\
        D. \lstinline|data[data.capacity() - 1];| \\
        E. \lstinline|*data.end();| \\
    \end{tabular}
\begin{visibleenv}<2>
\iftoggle{heldback}{}{
\\ \myemph{A or C} \\
 or \lstinline|data[data.size() - 1]| \\ or
    \lstinline|*(data.end() - 1);|
}
\end{visibleenv}
\end{frame}

\begin{frame}{implementing with stack with arrays}
    \begin{itemize}
    \item track \textit{index} of top
    \item push(X): increment top index, place $X$ in index in array
    \item top(): read from array at top index
    \item pop(): decrement top index
    \item plus special case for full array
    \end{itemize}
\end{frame}
