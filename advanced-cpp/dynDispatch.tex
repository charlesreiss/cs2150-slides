\begin{frame}[fragile,label=staticDispatchPrint]{C++ defaults to static dispatch (1)}
\lstset{language=C++,basicstyle=\sourcecodepro\fontsize{10.5}{11}\selectfont,
    moredelim={**[is][\btHL<all:2>]{@2}{2@}},
    }
\begin{lstlisting}
class Parent {
public:
    void print() { cout << "Parent::print()\n"; }
};
class Child : public Parent {
public:
    @2void print() { cout << "Child::print()\n"; }2@
};
Parent* getParent() { return new Child; }
int main() {
    Parent *p = getParent();
    p->print();
    delete p;
}
\end{lstlisting}
\begin{tikzpicture}[overlay,remember picture]
    \node[draw=red, very thick, fill=white,anchor=east] at ([yshift=-3cm]current page.east) {
\begin{minipage}{6cm}
output:
\begin{Verbatim}
Parent::print()
\end{Verbatim}
\end{minipage}
};
\end{tikzpicture}
\end{frame}

\begin{frame}{static versus dynamic dispatch}
    \begin{itemize}
        \item static dispatch --- call method based on compile-time type
        \item dynamic dispatch --- call method based on run-time type
    \end{itemize}
\end{frame}

\begin{frame}[fragile,label=staticDispatchDestroy]{C++ defaults to static dispatch (2)}
\lstset{
    language=C++,basicstyle=\sourcecodepro\fontsize{10.5}{11}\selectfont,
    moredelim={**[is][\btHL<all:2>]{@2}{2@}},
}
\begin{lstlisting}
class Parent {
public:
    Parent() {cout << "Parent()\n"; }
    ~Parent() { cout << "~Parent()\n"; }
};
class Child : public Parent {
public:
    Child() { cout << "Child()\n"; }
    @2~Child() { cout << "~Child()\n"; }2@
};
Parent* getParent() { return new Child; }
int main() {
    Parent *p = getParent();
    delete p;
}
\end{lstlisting}
\begin{tikzpicture}[overlay,remember picture]
    \node[draw=red, very thick, fill=white,anchor=east] at ([yshift=-3cm]current page.east) {
\begin{minipage}{4cm}
    output (\textit{probably}):
\begin{Verbatim}
Parent()
Child()
~Parent()
\end{Verbatim}
\end{minipage}
};
\end{tikzpicture}
\end{frame}

\begin{frame}{virtual: ask for dynamic dispatch}
    \begin{itemize}
    \item \texttt{virtual} keyword --- ask for dynamic dispatch
    \item not default --- because slower:
        \begin{itemize}
        \item static dispatch: just a function call
        \item dynamic dispatch: lookup correct function first!
        \end{itemize}
    \end{itemize}
\end{frame}

\begin{frame}[fragile,label=dynamicDispatchPrint]{virtual methods (1)}
\lstset{language=C++,basicstyle=\sourcecodepro\fontsize{10.5}{11}\selectfont,
    moredelim={**[is][\btHL<all:1>]{@1}{1@}},
    }
\begin{lstlisting}
class Parent {
public:
    @1virtual1@ void print() { cout << "Parent::print()\n"; }
};
class Child : public Parent {
public:
    void print() { cout << "Child::print()\n"; }
};
Parent* getParent() { return new Child; }
int main() {
    Parent *p = getParent();
    p->print();
    delete p;
}
\end{lstlisting}
\begin{tikzpicture}[overlay,remember picture]
    \node[draw=red, very thick, fill=white,anchor=east] at ([yshift=-3cm]current page.east) {
\begin{minipage}{6cm}
output:
\begin{Verbatim}
Child::print()
\end{Verbatim}
\end{minipage}
};
\end{tikzpicture}
\end{frame}

\begin{frame}[fragile,label=dynDispatchDestroy]{virtual methods (1)}
\lstset{language=C++,basicstyle=\sourcecodepro\fontsize{10.5}{11}\selectfont,
    moredelim={**[is][\btHL<all:1>]{@1}{1@}},
    }
\begin{lstlisting}
class Parent {
public:
    Parent() {cout << "Parent()\n"; }
    @1virtual1@ ~Parent() { cout << "~Parent()\n"; }
};
class Child : public Parent {
public:
    Child() { cout << "Child()\n"; }
    ~Child() { cout << "~Child()\n"; }
};
Parent* getParent() { return new Child; }
int main() {
    Parent *p = getParent();
    delete p;
}
\end{lstlisting}
\begin{tikzpicture}[overlay,remember picture]
    \node[draw=red, very thick, fill=white,anchor=east] at ([yshift=-3cm]current page.east) {
\begin{minipage}{6cm}
output:
\begin{Verbatim}
Parent()
Child()
~Child()
~Parent()
\end{Verbatim}
\end{minipage}
};
\end{tikzpicture}
\end{frame}

\begin{frame}{virtual destructors}
    \begin{itemize}
    \item \myemph{required} if you call \texttt{delete} on a base-class pointer
        \begin{itemize}
        \item (but it's actually an instance of the subclass)
        \end{itemize}
    \item (even if destructor doesn't do anything)
        \begin{itemize}
            \item compiler might use destructor to know how much to free up
            \item C++ standard quote: ``If the static type of the object to be deleted is different from its
dynamic type, the static type shall be a base class of the dynamic type of the object to be deleted and the
static type shall have a virtual destructor or the behavior is undefined.''
        \end{itemize}
    \end{itemize}
\end{frame}
