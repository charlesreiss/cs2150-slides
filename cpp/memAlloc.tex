\begin{frame}[fragile,label=noNewObjects]{C++ local variables (1)}
\lstset{language=C++,style=small}
\begin{lstlisting}
Rational getTwoThirds() {
    Rational twoThirds(2, 3);
    return twoThirds;
}
\end{lstlisting}
    \begin{itemize}
        \item two thirds is \myemph{copied when function returns}
    \end{itemize}
\end{frame}

\begin{frame}[fragile,label=noNewObjects2]{C++ local variables (2)}
\lstset{language=C++,style=small}
\begin{lstlisting}
HugeValue computeHugeInteger() {
    HugeValue theHugeNumber = ...;
    return theHugeNumber;
}
\end{lstlisting}
    \begin{itemize}
        \item copy huge number --- very inefficiect?
    \end{itemize}
\end{frame}

\begin{frame}[fragile,label=pointerToLocal]{C++: pointer to local variables?}
\lstset{language=C++,style=small}
\begin{lstlisting}
Rational *brokenGetTwoThirds() {
    Rational twoThirds(2, 3);
    return &twoThirds;  // ERROR
}
\end{lstlisting}
\begin{itemize}
    \item twoThirds \myemph{no longer exists} when function returns
\item address likely to be reused for something else
\end{itemize}
\end{frame}

\begin{frame}[fragile,label=cppNewLocal]{new in C++}
\lstset{
    language=C++,
    style=smaller,
}
\begin{lstlisting}
Rational *getTwoThirds() {
    Rational *twoThirdsPointer = new Rational(2, 3);
    return twoThirdsPointer;
}
HugeValue *computeHugeNumber() {
    HugeValue *theHugeNumber = new HugeValue;
    ... /* set *theHugeNumber */ ...
    return theHugeNumber;
}
\end{lstlisting}
    \begin{itemize}
        \item does not copy --- returns a pointer
        \item new allocates space somewhere
    \end{itemize}
\end{frame}

\begin{frame}[fragile,label=cppNewDelete]{need for delete (1)}
\lstset{
    language=C++,
    style=smaller,
}
\begin{lstlisting}
Rational *getTwoThirds() {
    Rational *twoThirdsPointer;
    twoThirdsPointer = new Rational(2, 3);
    return twoThirdsPointer;
}

void showTwoThirds() {
    Rational *twoThirdsPointer = getTwoThirds();
    twoThirdsPointer->print();
}
\end{lstlisting}
\begin{itemize}
    \item what happens to where \texttt{twoThirdsPointer} points?
    \item<2> memory \myemph{remains used and allocated}
    \item<2> ``memory leak''
\end{itemize}
\end{frame}

\begin{frame}[fragile,label=cppNewDelete2]{need for delete (2)}
\lstset{
    language=C++,
    style=smaller,
    moredelim={**[is][\btHL<all:1>]{@1}{1@}},
    moredelim={**[is][\btHL<all:2-3>]{@2}{2@}},
}
\begin{lstlisting}
Rational *getTwoThirds() {
    Rational *twoThirdsPointer = new Rational(2, 3);
    return twoThirdsPointer;
}

void showTwoThirds() {
    Rational *twoThirdsPointer = getTwoThirds();
    @1twoThirdsPointer->print();1@
}

int main() { showTwoThirds(); @2aThing();2@ return 0; }
\end{lstlisting}
\begin{tikzpicture}
\tikzset{
    box/.style={draw,thick,font=\tt},
    dealloced/.style={alt=<2->{opacity=0.5}},
    stillAlloced/.style={alt=<3>{red,ultra thick}},
    >=Latex,
}
\node[box,label={[font=\small,dealloced]local variable},dealloced] (twoThirdsPointer) {twoThirdsPointer};
\node[box,stillAlloced,label={[stillAlloced,font=\small]allocated with new},right=1cm of twoThirdsPointer] (twoThirds) {twoThirds};
\draw[thick,->,dealloced] (twoThirdsPointer) -- (twoThirds);

\end{tikzpicture}
\end{frame}

\begin{frame}[fragile,label=cppNewDelete3]{fixed example}
\lstset{
    language=C++,
    style=smaller,
    moredelim={**[is][\btHL<all:1>]{@1}{1@}},
    moredelim={**[is][\btHL<all:2-3>]{@2}{2@}},
}
\begin{lstlisting}
Rational *getTwoThirds() {
    Rational *twoThirdsPointer = new Rational(2, 3);
    return twoThirdsPointer;
}

void showTwoThirds() {
    Rational *twoThirdsPointer = getTwoThirds();
    twoThirdsPointer->print();
    @1delete twoThirdsPointer;1@
    // accessing twoThirdsPointer is now an ERROR
}
\end{lstlisting}
\begin{visibleenv}<2->
an error --- but may or may not crash (!) \\
whatever ends up at same address
\end{visibleenv}
\end{frame}


\begin{frame}[fragile,label=cppArray]{C++: fixed-sized arrays}
\lstset{language=C++,style=small,
    moredelim={**[is][\btHL<all:1>]{@2}{2@}},
    }
\begin{lstlisting}
@2int arrayOfTenValues[10]2@;
...
int fourthValue = arrayOfTenValues[3];
arrayOfTenValues[5] = newSixthValue;
\end{lstlisting}
\end{frame}

\begin{frame}[fragile,label=vlas]{C++: variable sized arrays?}
\lstset{language=C++,style=small,
    moredelim={**[is][\btHL<all:1>]{@1}{1@}},
    }
\begin{lstlisting}
int n;
cout << "Enter size: ";
cin >> n;
...
int brokenArrayOfNValues@1[n]1@;
...
\end{lstlisting}
\myemph{not part of C++} \\
(but some compilers allow an extension)
\begin{Verbatim}[fontsize=\small]
$ clang++ -Wall -pedantic -c test.cpp
test.cpp:3:29: warning: variable length arrays are a C99 feature [-Wvla-extension]
    int brokenArrayOfNValues[n];
\end{Verbatim}
\end{frame}


\begin{frame}[fragile,label=dynArray1]{C++: dynamic arrays (1)}
\lstset{
    language=C++,
    style=smaller,
    moredelim={**[is][\btHL<all:2>]{@2}{2@}},
}
\begin{lstlisting}
int n;
cout << "Enter size: ";
cin >> n;

// use the user's input to create an array of int
int * ages = @2new int [n];2@
\end{lstlisting}
\hrule
\begin{tikzpicture}
\tikzset{
    codeBox/.style={draw,thick},
    pointerBox/.style={draw},
    >=Latex,
    pointing/.style={->,ultra thick},
    value/.style={green!50!black},
    pointer/.style={blue},
}
\matrix[tight matrix,
    nodes={minimum height=.6cm},
    column 1/.style={nodes={draw=none,text width=4cm}},
    column 2/.style={nodes={text width=2cm,font=\tt}},
    row 1/.style={nodes={font=\normalfont,draw=none}},
    anchor=north west,
] (mem) {
    address \& value \\
    |[pointer]| 10000 \& |[pointer,alias=ages]| 90000 \\
    \ldots \& \ldots \\
    |[value]| 90000 \& |[value,alias=agesZero]| ? \\
    |[value]| 90004 \& |[value,alias=agesOne]| ? \\
    |[value]| 90008 \& |[value,alias=agesTwo]| ? \\
    \ldots \& \ldots \\
    |[value]| 90000+(n-1)$\times$4 \& |[value,alias=agesN1]| ? \\
};
    \node[font=\tt\small,pointer,right=0.25cm of ages] {ages};
    \node[font=\tt\small,value,right=0.25cm of agesZero] {ages[0]};
    \node[font=\tt\small,value,right=0.25cm of agesOne] {ages[1]};
    \node[font=\tt\small,value,right=0.25cm of agesTwo] {ages[2]};
    \node[font=\tt\small,value,right=0.25cm of agesN1] {ages[n-1]};
\end{tikzpicture}
\end{frame}

\begin{frame}[fragile,label=dynArray2]{C++: dynamic arrays (2)}
\lstset{
    language=C++,
    style=small,
    moredelim={**[is][\btHL<all:2>]{@2}{2@}},
}
\begin{lstlisting}
int * ages = @2new int [n];2@
... /* use ages[i] */ ...
@2delete[] ages;2@
\end{lstlisting}
\hrule
\begin{itemize}
    \item must \myemph{explicitly} free memory \ldots
    \item \ldots otherwise, remains allocated (until program exits)
    \item ``memory leak''
\end{itemize}
\end{frame}

\begin{frame}[fragile,label=dynArray3]{C++: dynamic arrays (3)}
\lstset{
    language=C++,
    style=small,
    moredelim={**[is][\btHL<all:2>]{@2}{2@}},
    moredelim={**[is][\btHL<all:2>]{@3}{3@}},
}
\begin{lstlisting}
int * ages = @2new int [n]2@;
for (int i = 0; i < n; i++) {
    cout << "Value for ages[" << i << "]: ";
    cin >> @3ages[i]3@;
}
for (int i = 0; i < n; i++)
    cout << "ages[" << i << "] = " << @3ages[i]3@
         << endl;
@2delete[] ages2@;
\end{lstlisting}
\end{frame}

\begin{frame}[fragile,label=newDelete]{new/delete}
\lstset{
    language=C++,
    style=smaller,
    moredelim={**[is][\btHL<all:2>]{@2}{2@}},
    moredelim={**[is][\btHL<all:3>]{@3}{3@}},
}
\begin{lstlisting}
// single integer
int *p;         p = new int;            delete p;
int *p;         p = @3new int(3)3@;         delete p;

// array of integers
int *p;         p = new int@2[2@100@2]2@;       @2delete[]2@ p;

Rational *p;    p = new Rational;       delete p;
Rational *p;    p = @3new Rational(3,4)3@;  delete p;
\end{lstlisting}
\begin{tikzpicture}[overlay,remember picture]
\coordinate (place) at ([yshift=2cm]current page.south);
\tikzset{
    markBox/.style={draw=red,very thick,align=left,at=(place)},
}
\begin{visibleenv}<2>
    \node[markBox] {\texttt{delete[]} form needed for new with arrays \\
                    idea: size information must be stored for arrays, \\
                    but single values};
\end{visibleenv}
\begin{visibleenv}<3>
    \node[markBox] {\texttt{new \textit{TYPE}(arg1, arg2)} --- calls constructor \\
                    built-in constructors for primitive types takes value to copy};
\end{visibleenv}
\end{tikzpicture}
\end{frame}
