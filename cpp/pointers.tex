% FIXME: questions about pointer code

\begin{frame}{pointers}
\begin{itemize}
\item store \myemph{memory addresses}
    \begin{itemize}
    \item the location of values
    \end{itemize}
\end{itemize}
\end{frame}

% FIXME: ways of interpreting?

\begin{frame}{memory?}
\begin{tikzpicture}
\matrix[tight matrix,
    column 1/.style={nodes={draw=none,text width=2cm}},
    column 2/.style={nodes={text width=2cm,font=\tt}},
    row 1/.style={nodes={font=\normalfont}},
    label={[font=\bfseries]north:memory (as 64-bit values)}
] (mem64) {
    address \& value (64-bit) \\
    0 \& 123999 \\
    8 \& 323232 \\
    16 \& 434093 \\
    \ldots \& \ldots \\
    10000 \& 1 \\
    10008 \& 5 \\
    10016 \& 7\\
    \ldots \& \ldots \\
};
\begin{visibleenv}<2>
\matrix[tight matrix,
    column 1/.style={nodes={draw=none,text width=2cm}},
    column 2/.style={nodes={text width=2cm,font=\tt}},
    row 1/.style={nodes={font=\normalfont,draw=none}},
    label={[font=\bfseries]north:(as 8-bit values)},
    anchor=north west,
] (mem8) at ([xshift=1.25cm]mem64.north east) {
    address \& value (8-bit) \\
    0 \& 95 \\
    1 \& 228 \\
    2 \& 1 \\
    3 \& 0 \\
    4 \& 0 \\
    5 \& 0 \\
    6 \& 0 \\
    7 \& 0 \\
    8 \& 160 \\
    9 \& 238 \\
    10 \& 4 \\
    11 \& \ldots \\
    \ldots \& \ldots \\
};
\tikzset{
    hiBox/.style={draw=red,inner sep=0.5mm,ultra thick}
}
\node[hiBox,fit=(mem64-2-2)] {};
\node[hiBox,fit=(mem8-2-2) (mem8-9-2)] {};
\draw[red,ultra thick] (mem64-2-2.north east) -- (mem8-2-2.north west);
\draw[red,ultra thick] (mem64-2-2.south east) -- (mem8-9-2.south west);
\end{visibleenv}
\end{tikzpicture}
\end{frame}

\begin{frame}[fragile,label=memoryValues]{values in memory}
\lstset{
    language=C++,
    style=smaller,
    moredelim={**[is][\btHL<all:1>]{@1}{1@}},
    moredelim={**[is][\btHL<all:2>]{@2}{2@}},
    moredelim={**[is][\btHL<all:3>]{@3}{3@}},
    moredelim={**[is][\btHL<all:4>]{@4}{4@}},
    moredelim={**[is][\btHL<all:4>]{@5}{5@}},
}
\begin{tikzpicture}[remember picture]
\tikzset{
    codeBox/.style={draw,thick},
    pointerBox/.style={draw},
    >=Latex,
    pointing/.style={->,ultra thick},
    value/.style={green!50!black},
    pointer/.style={blue},
},
\node[codeBox] (mainBox) {
\begin{lstlisting}
long aLong = 42;
int anInt = 43;
int anotherInt = 44;
\end{lstlisting}
};
\matrix[tight matrix,
    nodes={minimum height=.7cm},
    column 1/.style={nodes={draw=none,text width=1.75cm}},
    column 2/.style={nodes={text width=2cm,font=\tt}},
    row 1/.style={nodes={font=\normalfont,draw=none}},
    label={[font=\bfseries,xshift=2cm]north:memory (as 64-bit values)},
    anchor=north west,
] (mem) at ([yshift=-1cm]mainBox.south west) {
    address \& value \\
    \ldots \& \ldots \\
    |[value]| 10000 \& |[value,alias=aLong]| 42 \\
    |[value]| 10008 \& |[value,alias=theInts]| \begin{tabular}{l|l} 43 & 44 \end{tabular} \\
    10016 \& \ldots \\
    \ldots \& \ldots \\
};
\begin{visibleenv}<2->
    \node[font=\tt\small,value,right=0.25cm of aLong] {aLong};
    \node[font=\tt\small,value,right=0.25cm of theInts] (theIntsLabel) {anInt, anotherInt};
\end{visibleenv}
\begin{visibleenv}<3->
\matrix[tight matrix,value,nodes={minimum height=.6cm},
    column 1/.style={nodes={draw=none,text width=1.5cm}},
    column 2/.style={nodes={text width=1cm,font=\tt}},
    right=0.25cm of theIntsLabel,
    ] {
    10008 \& 43 \\
    10012 \& 44 \\
};
\end{visibleenv}
\begin{visibleenv}<4>
    \node[right=1cm of mainBox,draw=red,thick,align=left] {
        all variables kept \myemph{in memory} \\
        (array of bytes where \\
        `everything' is stored)
    };
\end{visibleenv}
\end{tikzpicture}
\end{frame}


\begin{frame}[fragile,label=pointersIntro]{pointers}
\lstset{
    language=C++,
    style=smaller,
    moredelim={**[is][\btHL<all:1>]{@1}{1@}},
    moredelim={**[is][\btHL<all:2>]{@2}{2@}},
    moredelim={**[is][\btHL<all:3>]{@3}{3@}},
    moredelim={**[is][\btHL<all:4>]{@4}{4@}},
    moredelim={**[is][\btHL<all:4>]{@5}{5@}},
}
\begin{tikzpicture}
\tikzset{
    codeBox/.style={draw,thick},
    pointerBox/.style={draw},
    >=Latex,
    pointing/.style={->,ultra thick},
    value/.style={green!50!black},
    pointer/.style={blue},
    infoBox/.style={draw=red,thick,align=left},
},
\node[codeBox] (mainBox) {
\begin{lstlisting}
@1long anInteger;1@
@1long *pointerToAnInteger;1@
anInteger = 42;
@2pointerToAnInteger = &anInteger;2@
@4*pointerToAnInteger = 43;4@
@5cout << pointerToInteger;5@
     // output: address (10000)
        // lab machines: in hexadecimal
@5cout << *pointerToInteger;5@
     // output: 43
\end{lstlisting}
};
\matrix[tight matrix,
    column 1/.style={nodes={draw=none,text width=1.75cm}},
    column 2/.style={nodes={text width=2cm,font=\tt}},
    row 1/.style={nodes={font=\normalfont,draw=none}},
    label={[font=\bfseries,xshift=1cm]north:memory (as 64-bit values)},
    anchor=north west,
] (mem) at ([yshift=-1cm]mainBox.south west){
    address \& value \\
    \ldots \& \ldots \\
    |[value]| 10000 \& |[value,alias=anInteger]| \sout<4->{42}\only<4->{~43} \\
    |[pointer]| 10008 \& |[pointer,alias=pointerToAnInteger]| \only<1-2>{?}\only<3->{10000} \\
    10016 \& \ldots \\
    \ldots \& \ldots \\
};
\begin{visibleenv}<2>
    \node[right=.5cm of mainBox,infoBox] {
        \&: ``address of''
    };
\end{visibleenv}

\begin{visibleenv}<3>
    \node[right=.5cm of mainBox,infoBox] {
       *: ``dereference'' \\ use value \\ at address
    };
\end{visibleenv}
\node[font=\tt\small,value,right=0.5cm of anInteger] (anIntegerName) {anInteger};
\node[font=\tt\small,pointer,right=.5cm of pointerToAnInteger] {pointerToAnInteger};
\begin{visibleenv}<3->
\draw[pointing] ([xshift=-.5cm]pointerToAnInteger.east) -- ++ (1cm,0cm) |- (anInteger.east);
\node[font=\tt\small,value,right=0.25cm of anIntegerName] {*pointerToAnInteger};
\end{visibleenv}
\end{tikzpicture}
\end{frame}


% FIXME: not about whitespace
\begin{frame}[fragile,label=pointerDeclare]{declaring pointers}
\lstset{
    language=C++,
    style=smaller,
    morekeywords=Rational,
}
\begin{lstlisting}
float *X; // X is a pointer to float
float* X; // X is a pointer to float
float * X; // X is a pointer to float

Rational *Y; // Y is a pointer to Rational
Rational* Y; // Y is a pointer to Rational

Rational **Z; // Z is a pointer to pointer to Rational
\end{lstlisting}
\end{frame}

% FIXME: not about whitespace
\begin{frame}[fragile,label=pointerDeclareMult]{declaring multiple pointers}
\lstset{
    language=C++,
    style=small,
    morekeywords=Rational,
}
\begin{lstlisting}
float *X, *Y; // X and Y are pointers to float
float *Z, ThisIsProbablyAMistake;
    // Z is a pointer to float
    // ThisIsProbablyAMistake is a float
\end{lstlisting}
\end{frame}

\begin{frame}<1-5>[fragile,label=pointOtherType]{pointers to other types}
\lstset{
    language=C++,
    style=smaller,
    moredelim={**[is][\btHL<all:1>]{@1}{1@}},
    moredelim={**[is][\btHL<all:2>]{@2}{2@}},
    moredelim={**[is][\btHL<all:3>]{@3}{3@}},
    moredelim={**[is][\btHL<all:4>]{@4}{4@}},
    moredelim={**[is][\btHL<all:4>]{@5}{5@}},
}
\begin{tikzpicture}
\tikzset{
    codeBox/.style={draw,thick},
    pointerBox/.style={draw},
    >=Latex,
    pointing/.style={->,ultra thick},
    value/.style={green!50!black},
    pointer/.style={blue},
    paren/.style={black!50,font=\small},
    infoBox/.style={draw=red,thick,align=left},
}
\node[codeBox] (mainBox) {
\begin{lstlisting}
Rational aFraction(2, 3);
Rational *pointerToFraction;
@4pointerToFraction = &aFraction4@;
@5*pointerToFraction5@ =
    (@5*pointerToFraction5@).times(@5*pointerToFraction5@);
\end{lstlisting}
};
\matrix[tight matrix,
    nodes={minimum height=.6cm},
    column 1/.style={nodes={draw=none,text width=1.75cm}},
    column 2/.style={nodes={text width=2cm,font=\tt}},
    row 1/.style={nodes={font=\normalfont,draw=none}},
    label={[font=\bfseries,xshift=1cm]north:memory},
    anchor=north west,
] (mem) at ([yshift=-1cm]mainBox.south west){
    address \& value \\
    \ldots \& \ldots \\
    |[value]| 10000 \& |[value,alias=fraction,font=\small\tt]| {\begin{tabular}{l|l} \sout<5->{2}~\only<5->{\myemph<5>{4}} & \sout<5->{3}~\only<5->{\myemph<5>{9}} \end{tabular}} \\
    |[pointer]| 10008 \& |[pointer,alias=pointerToFraction]| \only<1-2>{?}\only<3->{10000} \\
    10016 \& \ldots \\
    \ldots \& \ldots \\
};

\begin{visibleenv}<2->
\node[font=\tt\small,value,right=0.5cm of fraction] (fractionLabel) {aFraction};
\node[font=\tt\small,pointer,right=.5cm of pointerToFraction] {pointerToFraction};
\end{visibleenv}
\begin{visibleenv}<3>
\matrix[tight matrix,value,nodes={minimum height=.6cm},
    column 1/.style={nodes={draw=none,text width=1.75cm}},
    column 2/.style={nodes={text width=2cm,font=\tt}},
    right=1cm of fractionLabel,
    ] {
    10000 \& 2 \\
    10004 \& 3 \\
};
\end{visibleenv}
\begin{visibleenv}<4->
\draw[pointing] ([xshift=-.5cm]pointerToFraction.east) -- ++ (1cm,0cm) |- (fraction.east);
\node[font=\tt\small,value,right=0.25cm of fractionLabel] {*pointerToFraction};
\end{visibleenv}
\end{tikzpicture}
\end{frame}

\begin{frame}[fragile,label=dereference]{dereference operator}
    \begin{itemize}
        \item expression: \verb|*foo| is ``value pointed to by \verb|foo|''
        \begin{itemize}
            \item (different than declaration: \verb|Type *foo| means\\``foo is a pointer to Type'')
        \end{itemize}
        \vspace{.5cm}
        \item \lstinline|cout << *foo;| --- output value foo points to
        \item \lstinline|*foo = 42;| --- set value foo points to to 42
    \end{itemize}
\end{frame}

\begin{frame}[fragile,label=derefAndDecl]{dereference v declare}
\lstset{language=C++,style=small}
\begin{lstlisting}
int *pointer = &foo;
// same as:
int *pointer;
pointer = &foo;
\end{lstlisting}
\hrule
\begin{visibleenv}<2->
\begin{lstlisting}
int *pointer = &foo;
*pointer = bar;  // sets foo to bar
pointer = &bar;  // changes where pointer points
\end{lstlisting}
\end{visibleenv}
\end{frame}

\begin{frame}[fragile,label=addressOf]{address-of operator}
    \begin{itemize}
        \item in an expression: \verb|&foo| is ``\myemph{address of} \verb|foo|''
        \begin{itemize}
            \item (different than declaration: \verb|int &foo = 42;| means \\
                `foo is a \textit{reference}'' --- more on that later)
        \end{itemize}
        \vspace{.5cm}
        \item returns address of variable/value
            \begin{itemize}
            \item {\tt \&variable}, {\tt \&array[42]}, {\tt \&obj.instVar}
            \item error if applied to temporary values (e.g. \sout{\tt \&(2+2)})
            \end{itemize}
        \vspace{.5cm}
        \item \lstinline|cout << &foo;| --- output address of foo
        \item \lstinline|foo = &bar;| --- set {\tt foo} to be a pointer to {\tt bar}
    \end{itemize}
\end{frame}

% swap example

    % the -> operator
\againframe<6>{pointOtherType}

\begin{frame}[fragile,label=arrowOp]{\texttt{-\textgreater} operator}
\lstset{
    language=C++,
    style=small,
    moredelim={**[is][\btHL<all:1>]{@1}{1@}},
    moredelim={**[is][\btHL<all:2>]{@2}{2@}},
    moredelim={**[is][\btHL<all:3>]{@3}{3@}},
    moredelim={**[is][\btHL<all:4>]{@4}{4@}},
    moredelim={**[is][\btHL<all:4>]{@5}{5@}},
}
    \begin{itemize}
        \item \lstinline|(*foo).bar| same as \lstinline|foo->bar|
\begin{lstlisting}
Rational *pointerToFraction = &aFraction; 

aValue = pointerToFraction->times(
            *pointerToFraction);
// same as:
aValue = (*pointerToFraction).times(
            *pointerToFraction);
\end{lstlisting}
    \end{itemize}
\end{frame}

% FIXME: memory errors and NULL

\begin{frame}[fragile,label=NULL]{NULL}
\lstset{
    language=C++,
    style=smaller,
    moredelim={**[is][\btHL<all:1>]{@1}{1@}},
    moredelim={**[is][\btHL<all:2>]{@2}{2@}},
    moredelim={**[is][\btHL<all:3>]{@3}{3@}},
    moredelim={**[is][\btHL<all:4>]{@4}{4@}},
    moredelim={**[is][\btHL<all:4>]{@5}{5@}},
}
\begin{itemize}
\item \verb|NULL| or \verb|0| --- explicitly invalid pointer
    \begin{itemize}
    \item for {\tt NULL}: \lstinline|#include <cstddef>|, etc.
    \end{itemize}
\end{itemize}
\begin{lstlisting}
int anInt = 42;
int *pointer = @2NULL2@;
int *pointer = @202@; // same as above
// NOT same as: int *pointer;

*pointer = anInt;   // ERROR: crash (hopefully)
anInt = *pointer;   // ERROR: crash (hopefully)
pointer = anInt;    // ERROR: need cast

if (pointer == NULL) { ... }
if (!pointer) { ... } // same as above

if (pointer != NULL) { ... }
if (pointer) { ... } // same as above
\end{lstlisting}
\end{frame}

\begin{frame}[fragile,label=NullErrors]{crash (hopefully)}
\lstset{language=C++,style=small}
    \begin{itemize}
        \item Java --- using a null pointer triggers \texttt{NullPointerException}
        \item C++ --- using a null pointer \myemph{usually crashes}
            \begin{itemize}
            \item but not always --- not required
            \end{itemize}
    \end{itemize}
\end{frame}

\begin{frame}[fragile,label=uninitialized]{uninitialized values}
\lstset{language=C++,style=small}
    \begin{itemize}
        \item uninitialized pointers \myemph{are not always null}
            \begin{itemize}
            \item whatever was stored in that part of memory before
            \end{itemize}
        \item might crash or \\ might \myemph{silently point to something important}
    \end{itemize}
\end{frame}
% Binky pointer fun?

\begin{frame}[fragile,label=ptp]{pointer-to-pointers}
\tikzset{
    codeBox/.style={draw,thick},
    pointerBox/.style={draw},
    >=Latex,
    pointing/.style={->,ultra thick},
    value/.style={green!50!black},
    pointer/.style={blue},
}
\lstset{
    language=C++,
    style=smaller,
    moredelim={**[is][\btHL<all:1>]{@1}{1@}},
    moredelim={**[is][\btHL<all:2>]{@2}{2@}},
    moredelim={**[is][\btHL<all:3>]{@3}{3@}},
    moredelim={**[is][\btHL<all:4>]{@4}{4@}},
    moredelim={**[is][\btHL<all:5>]{@5}{5@}},
}
\begin{lstlisting}
int valueOne = 42, valueTwo = 100;
@2int *pointer = &valueOne;2@
@2int **ptrToPtr = &pointer;2@
@3**ptrToPtr -= 10;3@
@4*ptrToPtr = &valueTwo;4@
@5**ptrToPtr += 10;5@
// output: 32 110 110
cout << valueOne << " " << valueTwo << " "
     << *pointer << endl;
\end{lstlisting}
\begin{tikzpicture}
\matrix[tight matrix,
    column 1/.style={nodes={draw=none,text width=1.75cm}},
    column 2/.style={nodes={text width=4.5cm,font=\tt}},
    row 1/.style={nodes={font=\normalfont,draw=none}},
    anchor=north west,
] (mem) at ([yshift=-1cm]mainBox.south west){
    address \& value \\
    \ldots \& \ldots \\
    |[value]| 10000 \& |[value,alias=valueOne]| \sout<3->{42}\only<3->{\myemph<3>{~32}} \\
    |[value]| 10004 \& |[value,alias=valueTwo]| \sout<4->{100}\only<5->{\myemph<5>{~43}} \\
    |[pointer]| 10008 \& |[pointer,alias=pointer]| \sout<5->{10000}\only<4->{\myemph<4>{~10004}} \\
    |[pointer]| 10016 \& |[pointer,alias=ptrToPtr]| 10008 \\
    10024 \& \ldots \\
    \ldots \& \ldots \\
};
\begin{visibleenv}<2->
\node[font=\tt\small,value,right=0.5cm of valueOne] (v1) {valueOne};
\node[font=\tt\small,value,right=0.5cm of valueTwo] (v2) {valueTwo};
\node[font=\tt\small,pointer,right=.5cm of pointer] (ptr) {pointer};
\node[font=\tt\small,pointer,right=.5cm of ptrToPtr] (pp) {ptrToPtr};
\end{visibleenv}
\begin{visibleenv}<3>
\draw[pointing] ([xshift=-.5cm,yshift=.5mm]pointer.east) -- ++ (1cm,0cm) |- (valueOne.east);
\draw[pointing] ([xshift=-.5cm]ptrToPtr.east) -- ++ (1cm,0cm) |- ([yshift=-.5mm]pointer.east);
\end{visibleenv}
\begin{visibleenv}<4>
\draw[pointing,dotted,thick] ([xshift=-.5cm,yshift=.5mm]pointer.east) -- ++ (1cm,0cm) |- (valueOne.east);
\draw[pointing,red] ([xshift=-.5cm,yshift=.5mm]pointer.east) -- ++ (1cm,0cm) |- (valueTwo.east);
\draw[pointing] ([xshift=-.5cm]ptrToPtr.east) -- ++ (1cm,0cm) |- ([yshift=-.5mm]pointer.east);
\end{visibleenv}
\begin{visibleenv}<5>
\draw[pointing] ([xshift=-.5cm,yshift=.5mm]pointer.east) -- ++ (1cm,0cm) |- (valueTwo.east);
\draw[pointing] ([xshift=-.5cm]ptrToPtr.east) -- ++ (1cm,0cm) |- ([yshift=-.5mm]pointer.east);
\end{visibleenv}
\end{tikzpicture}
\end{frame}

\begin{frame}[fragile,label=swap]{swap}
\lstset{language=C++,style=smaller}
\begin{lstlisting}
void swap(Rational *a, Rational *b) {
    Rational temp = *a;
    *a = *b;
    *b = temp;
}

...
Rational first(4, 3);
Rational second(2, 7);
swap(&first, &second);
first.print();  // output: 2/7
\end{lstlisting}
\end{frame}

\begin{frame}[fragile,label=ptrQ]{pointer question}
\begin{lstlisting}
int a = 10, b = 20;
int *p; int *q;
p = &a;
q = p;
p = &b;
*p += 1;
*q = b;
\end{lstlisting}
What are the values of {\tt a}, {\tt b}? \\
\begin{tabular}{ll}
A. a=10, b=21 & D. a=21, b=21 \\
B. a=11, b=21 & E. something else \\
C. a=20, b=21 & F. possible crash \\
\end{tabular}
\end{frame}
