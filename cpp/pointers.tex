% FIXME: questions about pointer code

\begin{frame}{pointers}
\begin{itemize}
\item store \myemph{memory addresses}
    \begin{itemize}
    \item the location of values
    \end{itemize}
\end{itemize}
\end{frame}

% FIXME: ways of interpreting?

\begin{frame}{memory?}
\begin{tikzpicture}
\matrix[tight matrix,
    column 1/.style={nodes={draw=none,text width=2cm}},
    column 2/.style={nodes={text width=2cm,font=\tt}},
    row 1/.style={nodes={font=\normalfont}},
    label={[font=\bfseries]north:memory (as 64-bit values)}
] (mem64) {
    address \& value (64-bit) \\
    0 \& 123999 \\
    8 \& 323232 \\
    16 \& 434093 \\
    \ldots \& \ldots \\
    10000 \& 1 \\
    10008 \& 5 \\
    10016 \& 7\\
    \ldots \& \ldots \\
};
\begin{visibleenv}<2>
\matrix[tight matrix,
    column 1/.style={nodes={draw=none,text width=2cm}},
    column 2/.style={nodes={text width=2cm,font=\tt}},
    row 1/.style={nodes={font=\normalfont,draw=none}},
    label={[font=\bfseries]north:(as 8-bit values)},
    anchor=north west,
] (mem8) at ([xshift=1.25cm]mem64.north east) {
    address \& value (8-bit) \\
    0 \& 95 \\
    1 \& 228 \\
    2 \& 1 \\
    3 \& 0 \\
    4 \& 0 \\
    5 \& 0 \\
    6 \& 0 \\
    7 \& 0 \\
    8 \& 160 \\
    9 \& 238 \\
    10 \& 4 \\
    11 \& \ldots \\
    \ldots \& \ldots \\
};
\tikzset{
    hiBox/.style={draw=red,inner sep=0.5mm,ultra thick}
}
\node[hiBox,fit=(mem64-2-2)] {};
\node[hiBox,fit=(mem8-2-2) (mem8-9-2)] {};
\draw[red,ultra thick] (mem64-2-2.north east) -- (mem8-2-2.north west);
\draw[red,ultra thick] (mem64-2-2.south east) -- (mem8-9-2.south west);
\end{visibleenv}
\end{tikzpicture}
\end{frame}

\begin{frame}[fragile,label=memoryValues]{values in memory}
\lstset{
    language=C++,
    style=smaller,
    moredelim={**[is][\btHL<all:1>]{@1}{1@}},
    moredelim={**[is][\btHL<all:2>]{@2}{2@}},
    moredelim={**[is][\btHL<all:3>]{@3}{3@}},
    moredelim={**[is][\btHL<all:4>]{@4}{4@}},
    moredelim={**[is][\btHL<all:4>]{@5}{5@}},
}
\begin{tikzpicture}[remember picture]
\tikzset{
    codeBox/.style={draw,thick},
    pointerBox/.style={draw},
    >=Latex,
    pointing/.style={->,ultra thick},
    value/.style={green!50!black},
    pointer/.style={blue},
},
\node[codeBox] (mainBox) {
\begin{lstlisting}
long aLong = 42;
int anInt = 43;
int anotherInt = 44;
\end{lstlisting}
};
\matrix[tight matrix,
    nodes={minimum height=.7cm},
    column 1/.style={nodes={draw=none,text width=1.75cm}},
    column 2/.style={nodes={text width=2cm,font=\tt}},
    row 1/.style={nodes={font=\normalfont,draw=none}},
    label={[font=\bfseries,xshift=2cm]north:memory (as 64-bit values)},
    anchor=north west,
] (mem) at ([yshift=-1cm]mainBox.south west) {
    address \& value \\
    \ldots \& \ldots \\
    |[value]| 10000 \& |[value,alias=aLong]| 42 \\
    |[value]| 10008 \& |[value,alias=theInts]| \begin{tabular}{l|l} 43 & 44 \end{tabular} \\
    10016 \& \ldots \\
    \ldots \& \ldots \\
};
\begin{visibleenv}<2->
    \node[font=\tt\small,value,right=0.25cm of aLong] {aLong};
    \node[font=\tt\small,value,right=0.25cm of theInts] (theIntsLabel) {anInt, anotherInt};
\end{visibleenv}
\begin{visibleenv}<3->
\matrix[tight matrix,value,nodes={minimum height=.6cm},
    column 1/.style={nodes={draw=none,text width=1.5cm}},
    column 2/.style={nodes={text width=1cm,font=\tt}},
    right=0.25cm of theIntsLabel,
    ] {
    10008 \& 43 \\
    10012 \& 44 \\
};
\end{visibleenv}
\end{tikzpicture}
\end{frame}


\begin{frame}[fragile,label=pointersIntro]{pointers}
\lstset{
    language=C++,
    style=smaller,
    moredelim={**[is][\btHL<all:1>]{@1}{1@}},
    moredelim={**[is][\btHL<all:2>]{@2}{2@}},
    moredelim={**[is][\btHL<all:3>]{@3}{3@}},
    moredelim={**[is][\btHL<all:4>]{@4}{4@}},
    moredelim={**[is][\btHL<all:4>]{@5}{5@}},
}
\begin{tikzpicture}
\tikzset{
    codeBox/.style={draw,thick},
    pointerBox/.style={draw},
    >=Latex,
    pointing/.style={->,ultra thick},
    value/.style={green!50!black},
    pointer/.style={blue},
},
\node[codeBox] (mainBox) {
\begin{lstlisting}
long anInteger;
long *pointerToAnInteger;
anInteger = 42;
@2pointerToAnInteger = &anInteger;2@
@4*pointerToAnInteger = 43;4@
@5cout << pointerToInteger;5@ // output: (address, e.g. 10000)
@5cout << *pointerToInteger;5@ // output: 43
\end{lstlisting}
};
\matrix[tight matrix,
    column 1/.style={nodes={draw=none,text width=1.75cm}},
    column 2/.style={nodes={text width=2cm,font=\tt}},
    row 1/.style={nodes={font=\normalfont,draw=none}},
    label={[font=\bfseries,xshift=2cm]north:memory (as 64-bit values)},
    anchor=north west,
] (mem) at ([yshift=-1cm]mainBox.south west){
    address \& value \\
    \ldots \& \ldots \\
    |[value]| 10000 \& |[value,alias=anInteger]| \sout<4->{42}\only<4->{~43} \\
    |[pointer]| 10008 \& |[pointer,alias=pointerToAnInteger]| \only<1-2>{?}\only<3->{10000} \\
    10016 \& \ldots \\
    \ldots \& \ldots \\
};
\begin{visibleenv}<2->
\node[font=\tt\small,value,right=0.5cm of anInteger] (anIntegerName) {anInteger};
\node[font=\tt\small,pointer,right=.5cm of pointerToAnInteger] {pointerToAnInteger};
\end{visibleenv}
\begin{visibleenv}<3->
\draw[pointing] ([xshift=-.5cm]pointerToAnInteger.east) -- ++ (1cm,0cm) |- (anInteger.east);
\node[font=\tt\small,value,right=0.25cm of anIntegerName] {*pointerToAnInteger};
\end{visibleenv}
\end{tikzpicture}
\end{frame}

% FIXME: demonstration with cout


% FIXME: not about whitespace
\begin{frame}[fragile,label=pointerDeclare]{declaring pointers}
\lstset{
    language=C++,
    style=smaller,
    morekeywords=Rational,
}
\begin{lstlisting}
float *X; // X is a pointer to float
float* X; // X is a pointer to float
float * X; // X is a pointer to float

Rational *Y; // Y is a pointer to Rational
Rational* Y; // Y is a pointer to Rational

Rational **Z; // Z is a pointer to pointer to Rational
\end{lstlisting}
\end{frame}

\begin{frame}<1-5>[fragile,label=pointOtherType]{pointers to other types}
\lstset{
    language=C++,
    style=smaller,
    moredelim={**[is][\btHL<all:1>]{@1}{1@}},
    moredelim={**[is][\btHL<all:2>]{@2}{2@}},
    moredelim={**[is][\btHL<all:3>]{@3}{3@}},
    moredelim={**[is][\btHL<all:4>]{@4}{4@}},
    moredelim={**[is][\btHL<all:4>]{@5}{5@}},
}
\begin{tikzpicture}
\tikzset{
    codeBox/.style={draw,thick},
    pointerBox/.style={draw},
    >=Latex,
    pointing/.style={->,ultra thick},
    value/.style={green!50!black},
    pointer/.style={blue},
    paren/.style={black!50,font=\small},
}
\node[codeBox] (mainBox) {
\begin{lstlisting}
Rational aFraction(2, 3);
Rational *pointerToFraction;
@4pointerToFraction = &aFraction4@;
*pointerToFraction =
    (*pointerToFraction).times(*pointerToFraction);
\end{lstlisting}
};
\matrix[tight matrix,
    nodes={minimum height=.6cm},
    column 1/.style={nodes={draw=none,text width=1.75cm}},
    column 2/.style={nodes={text width=2cm,font=\tt}},
    row 1/.style={nodes={font=\normalfont,draw=none}},
    label={[font=\bfseries,xshift=2cm]north:memory},
    anchor=north west,
] (mem) at ([yshift=-1cm]mainBox.south west){
    address \& value \\
    \ldots \& \ldots \\
    |[value]| 10000 \& |[value,alias=fraction,font=\small\tt]| {\begin{tabular}{l|l} \sout<5->{2}~\only<5->{4} & \sout<5->{3}~\only<5->{9} \end{tabular}} \\
    |[pointer]| 10008 \& |[pointer,alias=pointerToFraction]| \only<1-2>{?}\only<3->{10000} \\
    10016 \& \ldots \\
    \ldots \& \ldots \\
};

\begin{visibleenv}<2->
\node[font=\tt\small,value,right=0.5cm of fraction] (fractionLabel) {aFraction};
\node[font=\tt\small,pointer,right=.5cm of pointerToFraction] {pointerToFraction};
\end{visibleenv}
\begin{visibleenv}<3>
\matrix[tight matrix,value,nodes={minimum height=.6cm},
    column 1/.style={nodes={draw=none,text width=1.75cm}},
    column 2/.style={nodes={text width=2cm,font=\tt}},
    right=1cm of fractionLabel,
    ] {
    10000 \& 2 \\
    10004 \& 3 \\
};
\end{visibleenv}
\begin{visibleenv}<4->
\draw[pointing] ([xshift=-.5cm]pointerToFraction.east) -- ++ (1cm,0cm) |- (fraction.east);
\node[font=\tt\small,value,right=0.25cm of fractionLabel] {*pointerToFraction};
\end{visibleenv}
\end{tikzpicture}
\end{frame}

\begin{frame}[fragile,label=dereference]{dereference operator}
    \begin{itemize}
        \item expression: \verb|*foo| is ``value pointed to by \verb|foo|''
        \item (declaration: \verb|Type *foo| means ``foo is a pointer to Type'')
        \vspace{.5cm}
        \item (declaration \myemph{mirrors} use)
        \vspace{.5cm}
        \item \lstinline|cout << *foo;| --- output value foo points to
        \item \lstinline|*foo = 42;| --- set value foo points to to 42
    \end{itemize}
\end{frame}

\begin{frame}[fragile,label=addressOf]{address-of operator}
    \begin{itemize}
        \item in an expression: \verb|&foo| is ``address of \verb|foo|''
        \item (in a declaration, e.g. \verb|int &foo = 42;| --- declares a ``reference'')
        \vspace{.5cm}
        \item takes any variable/expression, returns its address
        \vspace{.5cm}
        \item \lstinline|cout << &foo;| --- output address of foo
        \item \lstinline|foo = &bar;| --- set bar to be a pointer to foo
    \end{itemize}
\end{frame}


% FIXME: the & operator

% FIXME: the * operator

    % FIXME: relationship between declaration and other syntax

% swap example

    % the -> operator
\againframe<6>{pointOtherType}

\begin{frame}[fragile,label=arrowOp]{\texttt{-\textgreater} operator}
\lstset{
    language=C++,
    style=small
}
    \begin{itemize}
        \item \lstinline|(*foo).bar| same as \lstinline|foo->bar|
\begin{lstlisting}
... = pointerToFraction->times(
            *pointerToFraction)
\end{lstlisting}
    \end{itemize}
\end{frame}

% FIXME: memory errors and NULL

\begin{frame}[fragile,label=NULL]{NULL}
\lstset{language=C++,style=small}
    \begin{itemize}
    \item \verb|NULL| or \verb|0| --- explicitly invalid pointer
    \item similar to null in Java
\begin{lstlisting}
int anInt = 42;
int *pointer = NULL;
// same as: int *pointer = 0;

*pointer = anInt;   // ERROR: crash (hopefully)
anInt = *pointer;   // ERROR: crash (hopefully)
pointer = anInt;    // ERROR: type mismatch

if (pointer == NULL) { ... }
if (!pointer) { ... } // same as above

if (pointer != NULL) { ... }
if (pointer) { ... } // same as above
\end{lstlisting}
    \end{itemize}
\end{frame}

\begin{frame}[fragile,label=NullErrors]{crash (hopefully)}
\lstset{language=C++,style=small}
    \begin{itemize}
        \item Java --- using a null pointer triggers \texttt{NullPointerException}
        \item C++ --- using a null pointer \myemph{usually crashes}
            \begin{itemize}
            \item but not always --- not required
            \end{itemize}
    \end{itemize}
\end{frame}

\begin{frame}[fragile,label=uninitialized]{uninitialized values}
\lstset{language=C++,style=small}
    \begin{itemize}
        \item uninitialized pointers \myemph{are not always null}
            \begin{itemize}
            \item whatever was stored in that part of memory before
            \end{itemize}
        \item might crash or might \myemph{silently point to something important}
    \end{itemize}
\end{frame}
% Binky pointer fun?

% FIXME: pointers to pointers

% FIXME: swap example
\begin{frame}[fragile,label=swap]{swap}
\lstset{language=C++,style=smaller}
\begin{lstlisting}
void swap(Rational *a, Rational *b) {
    Rational temp = *a;
    b = *a;
    *b = temp;
}

...
Rational first(4, 3);
Rational second(2, 7);
swap(&first, &second);
first.print();  // output: 2/7
\end{lstlisting}
\end{frame}
