\begin{frame}[fragile,label=intCellJava]{Java: IntCell.java (1)}
\lstset{language=Java,style=smaller,
    moredelim={**[is][\btHL<all:1>]{@1}{1@}},
    moredelim={**[is][\btHL<all:2>]{@2}{2@}},
    moredelim={**[is][\btHL<all:3>]{@3}{3@}},
    moredelim={**[is][\btHL<all:4>]{@4}{4@}},
    moredelim={**[is][\btHL<all:5>]{@5}{5@}},
    moredelim={**[is][\btHL<all:6>]{@6}{6@}},
    moredelim={**[is][\btHL<all:7>]{@7}{7@}},
    moredelim={**[is][\btHL<all:8>]{@8}{8@}},
}
\begin{lstlisting}
public class IntCell {
    public @2IntCell()2@ { this(0); }

    public @2IntCell(int initialValue)2@ {
        storedValue = initialValue;
    }

    public @3int getValue()3@ {
        return storedValue;
    }

    public @3void setValue(int newValue)3@ {
        storedValue = newValue;
    }

    private int storedValue;
}
\end{lstlisting}
\end{frame}

\begin{frame}[fragile,label=cppIntCellThree]{C++ version: three files}
\begin{itemize}
\item {\tt IntCell.h} --- ``header file'' with declarations \myemph{only}
    \begin{itemize}
    \item {\tt \#include}d by both files below
    \end{itemize}
\item {\tt IntCell.cpp} --- implementation of class
\item {\tt TestIntCell.cpp} --- example {\tt main()} that uses class
\end{itemize}
\end{frame}

\begin{frame}<1>[fragile,label=intCellH]{IntCell.h}
\lstset{
    language=C++,
    style=smaller,
    moredelim={**[is][\btHL<all:1>]{@1}{1@}},
    moredelim={**[is][\btHL<all:2-3>]{@2}{2@}},
    moredelim={**[is][\btHL<all:4>]{@4}{4@}},
    moredelim={**[is][\btHL<all:5>]{@5}{5@}},
    moredelim={**[is][\btHL<all:6-7>]{@6}{6@}},
    moredelim={**[is][\btHL<all:8>]{@7}{7@}},
    moredelim={**[is][\btHL<all:9>]{@8}{8@}},
    moredelim={**[is][\btHL<all:10>]{@9}{9@}},
    moredelim={**[is][\btHL<all:11>]{@A}{A@}},
}
\begin{lstlisting}
#@2ifndef INTCELL_H2@
#@2define INTCELL_H2@
class IntCell {
  @4public:4@
    @5IntCell5@( int initialValue @6= 06@ );

    @7int getValue()7@ @8const8@;
    @7void setValue(int val)7@;

  private:
    @9int storedValue;9@
}@A;A@
#@2endif2@
\end{lstlisting}
\begin{tikzpicture}[overlay,remember picture]
\tikzset{
    explainBox/.style={draw=red,ultra thick,align=left,fill=white},
}
\coordinate (c) at (current page.center);
\coordinate (c2) at ([yshift=1cm]current page.center);
\coordinate (c3) at ([yshift=2cm]current page.center);
\coordinate (c4) at ([yshift=-2cm]current page.center);
\begin{visibleenv}<2-3>
\node[explainBox] at (c) {
    ``boilerplate'' \\
    used to keep preprocessor from including file twice \\
    (more on this later)
};
\end{visibleenv}
\begin{visibleenv}<4>
\node[explainBox] at (c) {
    everything after this is public \\
    until {\tt private:} \\
    (default is private)
};
\end{visibleenv}
\begin{visibleenv}<5>
\node[explainBox] at (c) {
    constructor declaration
};
\end{visibleenv}
\begin{visibleenv}<6>
\node[explainBox] at (c) {
    default argument \\
    must be part of declaration (not definition)
};
\end{visibleenv}
\begin{visibleenv}<7>
\node[explainBox] at (c4) {
    could have two explicit constructors, too: \\
    {\tt IntCell(); } \\
    {\tt IntCell(int initialValue); }
};
\end{visibleenv}
\begin{visibleenv}<8>
\node[explainBox] at (c3) {
    method declarations \\
    (official C++ name for methods: ``member functions'')
};
\end{visibleenv}
\begin{visibleenv}<9>
\node[explainBox] at (c3) {
    ``const'' \textit{after parenthesis} --- \\
    indicates method does not change object \\
    ({\tt this} is const --- enforced by compiler)
};
\end{visibleenv}
\begin{visibleenv}<10>
\node[explainBox] at (c) {
    instance variable \\
    (official C++ name: ``member variable'')
};
\end{visibleenv}
\begin{visibleenv}<11>
\node[explainBox] at (c) {
    semicolon is \myemph{required!}
};
\end{visibleenv}
\end{tikzpicture}
\end{frame}

\againframe<2,4->{intCellH}

\begin{frame}[fragile,label=intCellCpp]{IntCell.cpp}
\lstset{
    language=C++,
    style=smaller,
    moredelim={**[is][\btHL<all:2>]{@2}{2@}},
    moredelim={**[is][\btHL<all:3>]{@3}{3@}},
    moredelim={**[is][\btHL<all:4>]{@4}{4@}},
    moredelim={**[is][\btHL<all:5>]{@5}{5@}},
    moredelim={**[is][\btHL<all:6>]{@6}{6@}},
    moredelim={**[is][\btHL<all:7>]{@7}{7@}},
    moredelim={**[is][\btHL<all:8>]{@8}{8@}},
}
\begin{lstlisting}
#include "IntCell.h"

@2IntCell::2@IntCell( @3int initialValue3@ ) @4:4@
        @4storedValue( initialValue )4@ {
}

int @2IntCell::2@getValue() @5const5@ {
    return storedValue;
}

void @2IntCell::2@setValue( int val ) {
    storedValue = val;
}
\end{lstlisting}
\begin{tikzpicture}[overlay,remember picture]
\tikzset{
    explainBox/.style={draw=red,ultra thick,align=left,fill=white},
}
\coordinate (c) at (current page.center);
\coordinate (c2) at ([yshift=2cm]current page.center);
\coordinate (c3) at ([yshift=-2cm]current page.center);
\begin{visibleenv}<2>
\node[explainBox] at (c3) {
    all method declarations prefixed with ``ClassName::'' \\
    {\tt ::} seperates class/namespace names from \\
    names within the class/namespace
};
\end{visibleenv}
\begin{visibleenv}<3>
\node[explainBox] at (c) {
    declaration had ``\texttt{int initialValue \myemph{= 0}}'' \\
    not repeated in definition (doing so is \myemph{an error})
};
\end{visibleenv}
\begin{visibleenv}<4>
\node[explainBox] at (c) {
    special syntax for initializing member variables \\
    used to call constructors (otherwise --- default constructors used!)\\
    {\tt : variable1(value), variable2(anotherValue), \ldots}
};
\end{visibleenv}
\begin{visibleenv}<5>
\node[explainBox] at (c2) {
    \texttt{const} (method called on const object) \\
    defintion and declaration \\
    (repeated in case both {\tt const} and non-{\tt const} \\
    method  with same name, arguments)
};
\end{visibleenv}
\end{tikzpicture}
\end{frame}


\begin{frame}[fragile,label=testIntCellCpp]{TestIntCell.cpp}
\lstset{
    language=C++,
    style=smaller,
    moredelim={**[is][\btHL<all:2-3>]{@2}{2@}},
    moredelim={**[is][\btHL<all:4>]{@4}{4@}},
    moredelim={**[is][\btHL<all:5>]{@5}{5@}},
    moredelim={**[is][\btHL<all:6>]{@6}{6@}},
    moredelim={**[is][\btHL<all:7>]{@7}{7@}},
    moredelim={**[is][\btHL<all:8>]{@8}{8@}},
}
\begin{lstlisting}
#include <iostream>
#include "IntCell.h"
using namespace std;

int main( ) {
    @2IntCell m12@;
    @4IntCell m2( 37 )4@;
    // output: 0 37
    cout << m1.getValue( ) << " "
         << m2.getValue( ) << endl;
    @5m1 = m2;5@
    m2.setValue( 40 );
    // output: 37 40
    cout << m1.getValue( ) << " " 
         << m2.getValue( ) << endl;
    return 0;
}
\end{lstlisting}
\begin{tikzpicture}[overlay,remember picture]
\tikzset{
    explainBox/.style={draw=red,ultra thick,align=left,fill=white},
}
\coordinate (c) at (current page.center);
\coordinate (c2) at ([yshift=2cm]current page.center);
\begin{visibleenv}<2>
\node[explainBox] at (c) {
    not a reference --- \myemph{cannot be null} \\
    represents the object itself
};
\end{visibleenv}
\begin{visibleenv}<3>
\node[explainBox] at (c) {
    calls the \myemph{default constructor} \\ 
    \texttt{IntCell::IntCell()}
};
\end{visibleenv}
\begin{visibleenv}<4>
\node[explainBox] at (c2) {
    calls \texttt{IntCell(37)} constructor
};
\end{visibleenv}
\begin{visibleenv}<5>
\node[explainBox] at (c) {
    \myemph{copies} m2 into m1 \\
    like assigning each member variable \\
    C++ objects are \myemph{values} (not references)
};
\end{visibleenv}
\end{tikzpicture}
\end{frame}
