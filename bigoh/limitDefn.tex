\begin{frame}{limit-based definition}
\[\lim\myemph<2>{\sup}_{n\to\infty} \frac{f(n)}{g(n)} = X\]
if only if\ldots
\begin{itemize}
\item $X < \infty$: $f\in O(g)$
\item $X > 0$: $f \in \Omega(g)$
\item $0 < X < \infty$: $f \in \Theta(g)$
\item $X = 0$: $f \in o(g)$
\item $X = \infty$ (and $\lim\inf$): $f \in \omega(g)$
\end{itemize}
\end{frame}

\begin{frame}{lim sup?}
\begin{itemize}
\item $\lim\sup \frac{f(n)}{g(n)}$ --- ``limit superior''
    \begin{itemize}
        \item equal to normal $\lim$ if it is defined
    \end{itemize}
\item only care about upper bound
\item e.g. $n^2$ in  $f(n) = \begin{cases} 1 & n \text{ odd} \\n^2 & n \text{ even} \\ \end{cases}$
\vspace{.5cm}
\item usually glossed over (including in Bloomfield's/Floryan's slides from prior semesters)
\end{itemize}
\end{frame}
