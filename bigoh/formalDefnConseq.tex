\begin{frame}{aside: forall/exists}
    \begin{itemize}
        \item $\forall n > 0$: for all $n > 0$
        \item $\exists n < 0 $: there exists an $n < 0$
    \end{itemize}
\end{frame}

\begin{frame}<1>[label=defnConseq]{definition consequences}
    \begin{itemize}
        \item If $f \in O(h)$ and $g \not\in O(h)$, which are true?
        \item \myemph<2>{\sout<5->{1. $\forall m>0$, $f(m)<g(m)$}}
                \begin{itemize}
                \item for all $m$, $f$ is less than $g$
                \end{itemize}
        \item 2. $\exists m>0$, $f(m)<g(m)$
                \begin{itemize}
                \item there exists an $m$, so $f$ is less than $g$
                \end{itemize}
        \item 3. $\exists m_0>0, \forall m>m_0$, $f(m)<g(m)$
                \begin{itemize}
                \item there exists an $m_0$, so for all $m$ larger, $f$ is less than $g$
                \end{itemize}
        \item 4. 1 and 2
        \item 5. 2 and 3
        \item 6. 1 and 2 and 3
    \end{itemize}
\end{frame}

\againframe<2>{defnConseq}
\newcommand{\notimplies}{%
  \mathrel{{\ooalign{\hidewidth$\not\phantom{=}$\hidewidth\cr$\implies$}}}}

\begin{frame}{\fontsize{16}{17}\selectfont $f\in O(h), g\not\in O(h) \notimplies \forall m. f(m)<g(m)$}
    \begin{itemize}
    \item counterexample --- $f(n) = 5n$; $g(n) = n^3$; $h(n) = n^2$ \\
        \begin{itemize}
            \item $f\in O(h)$: $5n\le c n^2$ for all $n>n_0$ with $c = 6$, $n_0 = 2$
            \item $g\not\in O(h)$: $n^3\le c n^2$? use $n\approx cn_0$ as counterexample
        \end{itemize}
     \item $m=2$: $f(m) = 10 \not< g(m) = 8$
    \vspace{.5cm}
    \item<2-> intuition: big-oh ignores behavior for small $n$
    \end{itemize}
\end{frame}

\begin{frame}{$n^3\not\in O(n^2)$}
big-Oh definition requires:
\[n^3 \le c n^2 \text{ for all $n > n_0$}\]
choose any $c > 1$ and $n_0 > 1$, then \\
\[
    n=cn_0 \text{ is a counterexample}
\]
\[ n^3 = c^3n_0^3 = cn_0 (cn_0)^2 > c n^2 \]
contradicting the definition
\\
{\small (and for $c < 1$, use $n = n_0+1$, etc.)}
\end{frame}

\begin{frame}{ \fontsize{16}{17}\selectfont$f\in O(h), g\not\in O(h) \implies \exists m. f(m)<g(m)$}
    \begin{itemize}
        \item intuition: should be true for `big enough' $m$
        \item assume definition of big-Oh:
            \begin{itemize}
                \item $f \in O(h)$: $\forall n>n_0:\;f(n) \le c h(n)$ (for a $n_0, c > 0$)
                \item $g \not\in O(h)$: $\exists n > n_0:\;g(n) > c h(n)$ (for \textit{any} $n_0, c > 0$)
            \end{itemize}
        \item assume $f$'s $n_0$, $c$
        \item use the $n$ that must exist for $g$ (from definition)
    \end{itemize}
\end{frame}

\begin{frame}{ \fontsize{15}{16}\selectfont$f\in O(h), g\not\in O(h) \implies? \exists m_0\forall m>m_0. f(m) <g(m)$}
    \begin{itemize}
    \item intuitively, seems so $g$ must grow faster than $f$ --- for big $m$:
        \begin{itemize}
        \item $f(m) < c_1\cdot h(m)$
        \item $g(m) < c_2\cdot h(m)$ 
        \end{itemize}
    \item but some corner case counterexamples:
        \begin{itemize}
            \item $f(n) = n$
            \item $g(n) = \begin{cases} 1 & n \text{ odd} \\n^2 & n \text{ even} \\ \end{cases}$
            \item $h(n) = n$
        \end{itemize}
    \item true with additional restriction:
        \begin{itemize}
        \item $f$, $g$ monotonic ($g(n) \le g(n+1)$, etc.)
        \end{itemize}
    \end{itemize}
\end{frame}
