\begin{frame}{formal definitions}
    \begin{itemize}
    \item $f(n) \in O(g(n))$: \\
        \hspace{.5cm}there exists $c > 0$ and $n_0 > 0$ such that \\
        \hspace{.5cm}for all $n > n_0$, $\myemph<2>{f(n) \le c \cdot g(n)}$
    \vspace{.5cm}
    \item<2-> $f(n) \in \Omega(g(n))$: \\
        \hspace{.5cm}there exists $c > 0$ and $n_0 > 0$ such that \\
        \hspace{.5cm}for all $n > n_0$, $\myemph<2>{f(n) \ge c \cdot g(n)}$
    \vspace{.5cm}
    \item<3-> $f(n) \in \Theta(g(n))$: \\
          \hspace{.5cm} $f(n) \in O(g(n))$ \myemph{and} $f(n) \in \Omega(g(n))$
    \end{itemize}
\end{frame}

\begin{frame}{formal definition example (1)}
    \begin{itemize}
    \item $f(n) \in O(g(n))$ if and only if \\
        \hspace{.5cm}there exists $c > 0$ and $n_0 > 0$ such that \\
        \hspace{.5cm}$f(n) \le c \cdot g(n)$ for all $n > n_0$
    \item Is $ n \in O(n^2)$:
        \begin{itemize}
        \item<2-> choose $c = 1$, $n_0 = 2$
        \item<2-> for $n > 2=n_0$: $n \le c\cdot n^2 = n^2$
        \item<2-> Yes!
        \end{itemize}
    \end{itemize}
\end{frame}

\begin{frame}{formal definition example (2)}
    \begin{itemize}
    \item $f(n) \in O(g(n))$ if and only if \\
        \hspace{.5cm}there exists $c > 0$ and $n_0 > 0$ such that \\
        \hspace{.5cm}$f(n) \le c \cdot g(n)$ for all $n > n_0$
    \item Is $10n \in O(n)$?
        \begin{itemize}
        \item<2-> choose $c = 11$\tikzmark{bigC}, $n_0 = 2$
        \item<2-> for $n > 2=n_0$: $f(n) = 10n \le c\cdot g(n) = 11n$
        \item<2-> Yes!
        \end{itemize}
    \end{itemize}
    \begin{tikzpicture}[overlay, remember picture]
        \begin{visibleenv}<3->
            \node[mycallout=bigC,anchor=north west] at ([xshift=-2cm,yshift=-1cm]pic cs:bigC) {
                don't need to choose smallest possible $c$
            };
        \end{visibleenv}
    \end{tikzpicture}
\end{frame}

\begin{frame}{negating formal definitions}
    \begin{itemize}
    \item $f \in O(g)$: there exists $c, n_0 > 0$ so for all $n > n_0$: $f(n) \le cg(n)$
    \item $f \not\in O(g)$:
        \begin{itemize}
            \item there does not exist $c, n_0 > 0$ so for all $n > n_0$: $f(n) \le cg(n)$
            \item for all $c, n_0$, there exists $n > n_0$: $f(n) > cg(n)$
        \end{itemize}
    \end{itemize}
\end{frame}


\begin{frame}{formal definition example (3)}
    \begin{itemize}
    \item $f(n) \in O(g(n))$ if and only if \\
        \hspace{.5cm}there exists $c > 0$ and $n_0 > 0$ such that \\
        \hspace{.5cm}$f(n) \le c \cdot g(n)$ for all $n > n_0$
    \item Is $n^2 \in O(n)$?
        \begin{itemize}
        \item<2-> no --- consider any $c, n_0 > 0$
        \item<2-> consider $n_{bad} = (c + 100)(n_0 + 100) > n_0$ \\
            $n_{bad}^2 = (c + 100)^2(n_0 + 100)^2 > c(c + 100)(n_0 + 100) = cn_{bad}$ \\
        \item<2-> so can't find $c, n_0$ that sastisfy definition
        \item<2-> (i.e. $f(n) = n_{bad}^2 \not\le c \cdot g(n_{bad}) = c n_{bad}$)
        \end{itemize}
    \item<3-> alternative
        \begin{itemize}
        \item<3-> $n_{bad} = \max\{c + 100, n_0 + 1\} > n_0$
        \end{itemize}
    \end{itemize}
\end{frame}


\begin{frame}{formal definition example (4)}
    \begin{itemize}
    \item $f(n) \in O(g(n))$ if and only if \\
        \hspace{.5cm}there exists $c > 0$ and $n_0 > 0$ such that \\
        \hspace{.5cm}$f(n) \le c \cdot g(n)$ for all $n > n_0$
    \item consider: $f(n) = 100\cdot n^2 + n$, $g(n) = n^2$:
        \begin{itemize}
        \item choose $c = 200$, $n_0 = 2$
        \item observe for $n > 2$: $100n^2 + n \le 101n^2$
        \item for $n > 2=n_0$: $f(n) = 100n^2 + n \le 101n^2 \le c\cdot g(n) = 200n^2$
        \end{itemize}
    \end{itemize}
\end{frame}

\begin{frame}{big-oh proofs generally}
    \begin{itemize}
    \item if proving yes case: 
        \begin{itemize}
        \item look at inequality
        \item \textit{choose} a large enough $c$ and $n_0$ that it's definitely true
        \item don't bother finding smallest $c$, $n_0$ that work
        \end{itemize}
    \item if proving no case:
        \begin{itemize}
        \item game: given $c, n_0$ find counter example
        \item general idea: \textit{choose} $n > n_0$ using a formula based on $c$
        \item show that this $n$ never satisfies the inequality
        \item don't bother showing it's true for all $n' > n$
        \item don't bother finding smallest $n$ that works
        \end{itemize}
    \end{itemize}
\end{frame}
