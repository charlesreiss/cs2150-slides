\begin{frame}{asymptotic growth rate or \textit{order}}
\begin{itemize}
\item compare two functions, but\ldots
\item ignore constant factors, small inputs
\item<2-> example: {\color{red!70!black}$f(n) = 1\,000\,000 \cdot n^2$}; {\color{blue!70!black}$g(n) = 2^n$}
    \begin{itemize}
    \item $g$ grows faster --- eventually much bigger than $f$
    \end{itemize}
\end{itemize}
\begin{tikzpicture}
\begin{visibleenv}<3->
\begin{axis}[width=12.5cm,height=5.5cm,samples=50,xmin=1,xmax=35,ymin=0,ymax=2e9]
\addplot[draw=red,very thick,domain=1:35] {1000000*x^2};
\addplot[draw=blue,very thick,domain=1:31] {2^x};
\begin{visibleenv}<4->
\fill[green,opacity=0.05] (axis cs:30,0) rectangle (axis cs:35,2147483648);
\draw[thick,-Latex] (axis cs:25,1.5e9) node[left] {goal: predict behavior here} -- (axis cs:31,1.5e9);
\end{visibleenv}
\begin{visibleenv}<5->
\fill[red,opacity=0.05] (axis cs:1,0) rectangle (axis cs:27,0.75e9);
\draw[thick,-Latex] (axis cs:10,0.8e9) node[above,inner sep=0.5mm] {ignore behavior here} -- (axis cs:10,0.5e9);
\end{visibleenv}
\end{axis}
\end{visibleenv}
\end{tikzpicture}
\end{frame}

\begin{frame}{preview: what functions?}
\begin{itemize}
    \item example: comparing sorting algorithms
    \item $\text{runtime} = f(\text{size of input})$
        \begin{itemize}
        \item e.g. $\text{seconds to sort} = f(\text{number of elements in list})$
        \item e.g. $\text{\# operations to sort} = f(\text{number of elements in list})$
        \end{itemize}
    \item $\text{space} = f(\text{size of input})$
        \begin{itemize}
        \item e.g. $\text{number of bytes of memory} = f(\text{number of elements in list})$
        \end{itemize}
\end{itemize}
\end{frame}

\begin{frame}{theory, not empirical}
    \begin{itemize}
    \item yes, you can make \textit{guesses} about big-oh behavior from actual measurements
    \item but, no, comparing graphs doesn't tell you big-oh
    \begin{itemize}
        \item what happens further to the right?
        \item asymptotic difference may only be \myemph{bigger than you can physically run}
    \end{itemize}
    \vspace{.5cm}
    \item want to write down \myemph{formula}
    \item<2-> example: summing a list of $n$ items:
        \begin{itemize}
        \item exactly $n$ addition operations
        \item \textit{assume} each one takes $k$ unit of time
        \item $\text{runtime} = f(n) = kn$
        \end{itemize}
    \end{itemize}
\end{frame}
