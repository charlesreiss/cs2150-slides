\usetikzlibrary{graphs,graphdrawing}
\usegdlibrary{trees}

\begin{frame}{expression tree and traversals}
\begin{tikzpicture}
\tikzset{
    >=Latex,
    myt/.style={binary tree layout,level distance=10mm,sibling distance=15mm,nodes={draw,circle,inner sep=0.5mm,minimum width=.8cm}},
}
\begin{scope}[myt]
\graph{
    [fresh nodes,name=g] top[as={+}] -> {
        "a",
        "*" -> {
            "+" -> {"b", "c"},
            "d"
        }
    }
};
\end{scope}
\node[align=left,anchor=north west] at ([xshift=3cm]g top) {
    \tt (a + ((b + c) * d)) \\
};
\end{tikzpicture}
\end{frame}

\begin{frame}{expression tree and traversals}
\begin{tikzpicture}
\tikzset{
    >=Latex,
    myt/.style={binary tree layout,level distance=10mm,sibling distance=15mm,nodes={draw,circle,inner sep=0.5mm,minimum width=.8cm}},
}
\begin{scope}[myt]
\graph{
    [fresh nodes,name=g] top[as={+}] -> {
        "a",
        "*" -> {
            "+" -> {"b", "c"},
            "d"
        }
    }
};
\end{scope}
\node[align=left,anchor=north west] at ([xshift=3cm]g top) {
    infix: \tt (a + ((b + c) * d)) \\
    postfix: \tt a b c + d * + \\
    prefix: \tt + a * + b c d
};
\end{tikzpicture}
\end{frame}

\begin{frame}{postfix expression to tree}
\begin{itemize}
\item use a stack \myemph{of trees}
\item number $n$ $\rightarrow$ push(\begin{tikzpicture}[baseline={([yshift=-.5ex]current bounding box.center)}]\node[draw,circle,inner sep=0.5mm]{$n$};\end{tikzpicture})
\item operator {\it OP} $\rightarrow$ 
\begin{itemize}
\item pop into $A$, $B$; then
\item push\begin{tikzpicture}[baseline={(current bounding box.center)}]
    \node[draw,circle, inner sep=0.5mm](op){\it OP};
    \node[below right=0.25cm of op,inner sep=.1mm] (a) {$A$};
    \node[below left=0.25cm of op,inner sep=.1mm] (b) {$B$};
    \draw[-Latex,thick] (op) -- (a);
    \draw[-Latex,thick] (op) -- (b);
    \end{tikzpicture}
\end{itemize}
\end{itemize}
\end{frame}

\begin{frame}[fragile,label=postFixToExprExample]{example}
\begin{tikzpicture}
\tikzset{
    >=Latex,
    myt/.style={binary tree layout,level distance=10mm,sibling distance=15mm,nodes={draw,circle,inner sep=0.5mm,minimum width=.8cm}},
    box/.style={draw,thick},
    line/.style={ultra thick,black!50},
    >=Latex,
}
\node[font=\tt] (eqn){\myemph<2>{a b} \myemph<3>{+} \myemph<4>{c d e} \myemph<5>{+} \myemph<6>{*} \myemph<7>*};
\coordinate (stackOne) at ([yshift=-6cm]eqn.south);
\coordinate (stackTwo) at ([yshift=-4cm]eqn.south);
\coordinate (stackThree) at ([yshift=-2.5cm]eqn.south);
\coordinate (stackFour) at ([yshift=-1cm]eqn.south);
\begin{visibleenv}<2-6>
\draw[line,->] ([xshift=2.2cm,yshift=-1cm]stackOne) -- ++(0cm,3cm) node[above right,inner sep=.5mm,black] {top of stack};
\end{visibleenv}
\begin{visibleenv}<2>
\begin{scope}[myt]
\graph{
    [fresh nodes,name=a1] a[desired at=(stackOne)]
};
\end{scope}
\begin{scope}[myt]
\graph{
    [fresh nodes,name=b1] b[desired at=(stackTwo)]
};
\end{scope}
\foreach \x in {stackOne,stackTwo} {
\draw[box] ([xshift=-2cm,yshift=-1cm]\x) rectangle ([xshift=2cm,yshift=1cm]\x);
}
\end{visibleenv}
\begin{visibleenv}<3-7>
\begin{scope}[myt]
\graph{
    [fresh nodes,name=p1] plus[as={+},desired at={([yshift=.5cm]stackOne)}] -> {a, b}
};
\end{scope}
\foreach \x in {stackOne} {
\draw[box] ([xshift=-2cm,yshift=-1cm]\x) rectangle ([xshift=2cm,yshift=1cm]\x);
}
\end{visibleenv}
\begin{visibleenv}<4>
\begin{scope}[myt]
\graph{
    [fresh nodes,name=c1] c[desired at=(stackTwo)]
};
\end{scope}

\begin{scope}[myt]
\graph{
    [fresh nodes,name=d1] d[desired at=(stackThree)]
};
\end{scope}
\begin{scope}[myt]
\graph{
    [fresh nodes,name=e1] e[desired at=(stackFour)]
};
\end{scope}
\foreach \x in {stackTwo} {
\draw[box] ([xshift=-2cm,yshift=-1cm]\x) rectangle ([xshift=2cm,yshift=.75cm]\x);
}
\foreach \x in {stackThree,stackFour} {
\draw[box] ([xshift=-2cm,yshift=-.75cm]\x) rectangle ([xshift=2cm,yshift=.75cm]\x);
}
\end{visibleenv}
\begin{visibleenv}<5>
\begin{scope}[myt]
\graph{
    [fresh nodes,name=c2] c[desired at=(stackTwo)]
};
\end{scope}
\begin{scope}[myt]
\graph{
    [fresh nodes,name=plus2] plus[desired at={([yshift=1cm]stackThree)},as={+}] -> {c, d}
};
\end{scope}
\foreach \x in {stackTwo} {
\draw[box] ([xshift=-2cm,yshift=-1cm]\x) rectangle ([xshift=2cm,yshift=.75cm]\x);
}
\draw[box] ([xshift=-2cm,yshift=-.75cm]stackThree) rectangle ([xshift=2cm,yshift=.25cm]stackFour);
\end{visibleenv}
\begin{visibleenv}<6-7>
\begin{scope}[myt]
\graph{
    [fresh nodes,name=times1] times[desired at={([xshift=-.5cm,yshift=2cm]stackTwo)},as={*}] -> {c, "+" -> {d, e}}
};
\end{scope}
\draw[box] ([xshift=-2cm,yshift=-1cm]stackTwo) rectangle ([xshift=2cm,yshift=0cm]stackFour);
\end{visibleenv}
\begin{visibleenv}<7>
\begin{scope}[myt]
\graph{
    [fresh nodes,name=times2] times[desired at={([xshift=8cm,yshift=3cm]stackTwo)},as={*}] -> {"+" -> {a, b},  "*" -> {c, "+" -> {d, e}}}
};
\end{scope}
    \draw[line width=.1cm,black!50,-Latex] ([xshift=-2cm]times2 a.west) -- ++(1.5cm,0cm);
\end{visibleenv}
\end{tikzpicture}
\end{frame}
