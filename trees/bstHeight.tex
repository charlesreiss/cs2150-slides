\usetikzlibrary{graphs,graphdrawing}
\usegdlibrary{trees}

\begin{frame}{binary tree: worst-case height}
\begin{tikzpicture}
\tikzset{
    >=Latex,
    mybst/.style={binary tree layout,level distance=10mm,sibling distance=15mm,nodes={draw,circle,inner sep=0.5mm,minimum width=.8cm}},
}
\begin{scope}[mybst]
\graph {
    [name=a] 1 -> 2[second,desired at={(0,0)}] -> 3[second] -> 4[second] -> 5[second] -> 6[second] -> 7[second]
};
\end{scope}
\node[align=left] at (-6, -1) {
    $n$-node BST: worst-case height/depth $n-1$ \\
};
\end{tikzpicture}
\end{frame}

\begin{frame}{binary tree: best-case height}
\begin{tikzpicture}
\tikzset{
    >=Latex,
    mybst/.style={binary tree layout,level distance=10mm,sibling distance=15mm,nodes={draw,circle,inner sep=0.5mm,minimum width=.8cm}},
}
\begin{scope}[mybst]
\graph {
    [name=a] 4[desired at={(0,0)}] -> {2 -> {1, 3}, 6 -> {5, 7}} 
};
\end{scope}
\node[align=left] at (0, 1) {
    height $h$: at most $2^{h+1}-1$ nodes
};
\end{tikzpicture}
\end{frame}

\begin{frame}{binary tree: proof best-case height is possible}
\begin{itemize}
\item proof \textbf{by induction}: can have $2^{h+1}-1$ nodes in $h$-height tree
\vspace{.25cm}
\item $\mathbf{h=0}$: $h=0$: exactly one node; $2^{h+1}-1=1$ nodes
    \vspace{.25cm}
\item $\mathbf{h=k\rightarrow h=k+1}$:
    \item start with \textit{two copies} of a maximum tree of height $k$
    \item create a new tree as follows:
        \begin{itemize}
        \item create a new root node
        \item add edges from the root node to the roots of the copies
        \end{itemize}
    \item the height of this new tree is $k+1$
        \begin{itemize}
        \item path of length $k$ in old tree + either new edge
        \end{itemize}
    \item the number of nodes is $2(2^{k+0}-1) + 1= 2^{k+1}-2 + 1 = 2^{k+1} - 1$
\end{itemize}
\end{frame}

\begin{frame}{binary tree: best-case height is best}
\begin{itemize}
\item (informally)
\item property of trees in root:
    \begin{itemize}
        \item except for the root, every node in tree has 2 children
    \end{itemize}
\item no way to add nodes without increasing height
    \begin{itemize}
    \item add below leaf --- longer path to root --- longer height
    \item add above root --- every old node has longer path to root
    \end{itemize}
\end{itemize}
\end{frame}


\begin{frame}{binary tree height formula}
    \begin{itemize}
    \item $n$: number of nodes
    \item $h$: height
    \end{itemize}
\begin{eqnarray*}
    n + 1 &\le& 2^{h+1} \\
    \log_2(n+1) &\le& \log_2\left(2^{h+1}\right) \\
    \log(n+1) &\le& h+1 \\
    h &\ge& \log_2\left(n+1\right)-1 \\
\end{eqnarray*}
    \begin{itemize}
    \item shortest tree of $n$ nodes: $\sim \log_2(n)$ height
    \end{itemize}
\end{frame}

\begin{frame}{perfect binary trees}
    \begin{tikzpicture}
\tikzset{
    >=Latex,
    mybst/.style={binary tree layout,level distance=10mm,sibling distance=15mm,nodes={draw,circle,inner sep=0.5mm,minimum width=.8cm,text=white}},
}
\begin{scope}[mybst]
\graph {
    [name=a] 4[desired at={(0,0)}] -> {2 -> {1, 3}, 6 -> {5, 7}} 
};
\end{scope}
\end{tikzpicture}
    \begin{itemize}
    \item a binary tree is \myemph{perfect} if
        \begin{itemize}
        \item all leaves have same depth
        \item all nodes have zero children (leaf) or two children
        \end{itemize}
    \item \myemph{exactly} the trees that achieve $2{h+1}-1$ nodes
    \end{itemize}
\end{frame}
