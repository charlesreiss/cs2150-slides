\usetikzlibrary{graphs,graphdrawing}
\usegdlibrary{layered}

\begin{frame}[fragile,label=progAsTree]{programs as trees}
\usemintedstyle{tango}
\begin{onlyenv}<1>
\begin{minted}[fontsize=\small]{C++}
int z;

int foo (int x) {
  for (int y = 0;
       y < x;
       y++)
    cout << y << endl;
}

int main() {
  int z = 5;
  cout << "enter x" << endl;
  cin >> z;
  foo(z);
}
\end{minted}
\end{onlyenv}
\begin{onlyenv}<2>
\begin{minted}[fontsize=\small,highlightlines={3-8}]{C++}
int z;

int foo (int x) {
  for (int y = 0;
       y < x;
       y++)
    cout << y << endl;
}

int main() {
  int z = 5;
  cout << "enter x" << endl;
  cin >> z;
  foo(z);
}
\end{minted}
\end{onlyenv}
\begin{tikzpicture}[overlay,remember picture]
\tikzset{
    >=Latex,
    myt/.style={layered layout,level distance=2mm,sibling distance=5mm,nodes={draw,ellipse,thin,inner sep=0.25mm,minimum width=1cm,align=center,font=\fontsize{9}{10}\selectfont,fill=white}},
    isFoo/.style={alt=<2>{draw=blue,fill=blue!10}},
}
\begin{scope}[myt]
\graph {
    [fresh nodes,name=g] program[alias=graphTop,desired at={([xshift=-7cm,yshift=-2cm]current page.north east)}] -> {
        vars -> "int z",
        functions -> {
            "foo()"[isFoo] -> {
                "params"[isFoo] -> "int z"[isFoo],
                "vars"[isFoo],
                "body"[isFoo] -> "for"[isFoo] -> {
                    "int y"[isFoo],
                    "y < x"[isFoo],
                    "y++"[isFoo],
                    "body"[isFoo] -> { "cout"[isFoo] -> {"y"[isFoo], endlBottom[as=endl,isFoo]}}
                },
            },
            main[as={main()}]  -> {
                "params",
                "vars" -> "int z" -> "=5",
                "body" [> minimum layers=6,alias=myBody] -> {
                    "cout" -> {
                        enterZ[as={\textquotedbl enter z\textquotedbl}],
                        "endl"
                    },
                    "cin" -> "z",
                    "foo()" -> zBottom[as={z}]
                }
            }
        }
    }
};
\end{scope}
\begin{pgfonlayer}{bg}
    \node[fill=white,fit=(g program) (g params) (g endlBottom)] {};
    \node[fill=white,fit=(g main) (g zBottom) (g enterZ)] {};
\end{pgfonlayer}
\end{tikzpicture}
\end{frame}

\begin{frame}[fragile,label=astExpl]{abstract syntax tree}
\usemintedstyle{tango}
\begin{tikzpicture}
\tikzset{
    >=Latex,
    myt/.style={layered layout,level distance=2mm,sibling distance=5mm,nodes={draw,ellipse,thin,inner sep=0.25mm,minimum width=1cm,align=center,font=\fontsize{10}{11}\selectfont,fill=white}},
    isFoo/.style={},
    box/.style={draw=blue,very thick,fill=white, align=left},
    forExample/.style={
        draw=blue,very thick,fill=blue!10,
    },
}
\begin{pgfonlayer}{fg}
\begin{visibleenv}<2-3>
\node[box,anchor=west] at ([xshift=.1cm,yshift=1cm]g for.east) {
    for loop: four children \\
    init, condition, update, body
};
\end{visibleenv}
\begin{visibleenv}<3>
\node[box,anchor=north west,align=left] at ([xshift=1.5cm]g for.east) {
\begin{minipage}{8cm}
\begin{minted}[fontsize=\fontsize{10}{11}\selectfont]{C++}
class ASTNode {
    ...
};

// public class ForNode extends ASTNode
class ForNode : public ASTNode {
    ...
private:
    ASTNode *init, *condition,
            *update, *body;
};
\end{minted}
\end{minipage}
};
\end{visibleenv}
\end{pgfonlayer}
\begin{scope}[myt]
\graph {
    [fresh nodes,name=g] program[alias=graphTop] -> {
        vars -> "int z",
        functions -> {
            "foo()"[isFoo] -> {
                "params"[isFoo] -> "int z"[isFoo],
                "vars"[isFoo],
                "body"[isFoo] -> "for"[isFoo,forExample] -> {
                    "int y"[isFoo,forExample],
                    "y < x"[isFoo,forExample],
                    "y++"[isFoo,forExample],
                    "body"[isFoo,forExample] -> { "cout"[isFoo] -> {"y"[isFoo], endlBottom[as=endl,isFoo]}}
                },
            },
            main[as={main()}]  -> {
                "params",
                "vars" -> "int z" -> "=5",
                "body" [alias=myBody] -> {
                    "cout" -> {
                        enterZ[as={\textquotedbl enter z\textquotedbl}],
                        "endl"
                    },
                    "cin" -> "z",
                    "foo()" -> zBottom[as={z}]
                }
            }
        }
    }
};
\end{scope}
\end{tikzpicture}
\end{frame}

\begin{frame}{AST applications}
\begin{itemize}
\item ``abstract syntax tree'' = ``parse tree''
\item part of how compilers work
\item do some tree traversal to do\ldots
    \begin{itemize}
        \item code generation --- e.g. \texttt{ASTNode::outputCode()} method
        \item optimization
        \item type checking\ldots
    \end{itemize}
\end{itemize}
\end{frame}

\begin{frame}{using AST to compare programs}
    \begin{itemize}
        \item comparing trees is a good way to compare programs\ldots
        \item while ignoring:
            \begin{itemize}
            \item function/method order (e.g. sort function nodes by length)
            \item variable names (e.g. ignore variable names when comparing)
            \item comments
            \item \ldots
            \end{itemize}
        \vspace{.5cm}
        \item part of many software plagerism/copy+paste detection tools
    \end{itemize}
\end{frame}
