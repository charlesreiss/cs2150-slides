\begin{frame}[fragile,label=mallocPrototype]{malloc}
\lstset{language=C++,style=small}
\begin{lstlisting}
void *malloc(size_t size);
\end{lstlisting}
\begin{itemize}
\item \texttt{size\_t} --- integer type that holds size (in bytes)
\end{itemize}
\end{frame}

\begin{frame}[fragile,label=mallocUsage1]{typical malloc usage}
\lstset{language=C++,style=smaller}
\begin{lstlisting}
int *array;
...
array = malloc(number_of_elements * sizeof(*array))
// OR
array = malloc(number_of_elements * sizeof(int))
\end{lstlisting}
\hrule
\begin{lstlisting}
SomeType *item;
...
item = malloc(sizeof(*item));
// OR
item = malloc(sizeof(SomeType));
\end{lstlisting}
\hrule
note: in C++ (not C) would need casts
    \begin{itemize}
        \item \lstinline|array = (int*) malloc(...);|
    \end{itemize}
\end{frame}

\begin{frame}{malloc and free}
\begin{itemize}
\item free --- undo malloc's allocation
\end{itemize}
\end{frame}

\begin{frame}[fragile,label=newParts]{new}
\lstset{language=C++}
\begin{itemize}
    \item new does \myemph{two things} that can be done seperately
    \vspace{.5cm}
    \item allocate memory
        \begin{itemize}
        \item \texttt{operator new(sizeof(Foo))}
        \end{itemize}
    \item call constructors
        \begin{itemize}
        \item can do separately with ``placement new''
        \item \lstinline|new (somePtr) Foo(arguments);|
        \end{itemize}
    \end{itemize}
\end{frame}

\begin{frame}[fragile,label=placementNew]{``manually'' doing what new does}
\lstset{language=C++}
\begin{lstlisting}
Foo *foo = new Foo(1, 2, 3);
\end{lstlisting}
\hrule
\begin{lstlisting}
#include <memory>  // prototypes for operator new
...

// allocate space
Foo *foo = (Foo*) operator new(sizeof(Foo));

// call constructor
new (foo) Foo(1, 2, 3);
\end{lstlisting}
\end{frame}

\begin{frame}[fragile,label=placementNewVector]{implementing vector: create}
\lstset{language=C++,style=smaller}
\begin{lstlisting}
template <class T> class MyVector {
    ...
private:
    T * array;
    int size, capacity;
};

template <class T>
void MyVector::push_back(const T& other) {
    // increase array capacity if needed
    if (++size > capacity) { ... }

    // call copy constructor to create array[size-1]
    new (&array[size - 1]) T(other);
        // better than constructing all in advance and assigning
        // e.g. if vector of lists,
        //      don't allocate "extra" head/tail dummy nodes
}
\end{lstlisting}
\end{frame}

\begin{frame}[fragile,label=deleteParts]{delete}
\lstset{language=C++}
\begin{itemize}
    \item delete does \myemph{two things} that can be done seperately
    \vspace{.5cm}
    \item call destructors
        \begin{itemize}
        \item \lstinline|foo->~Foo();|
        \end{itemize}
    \item actually free memory
        \begin{itemize}
        \item \lstinline|operator delete(foo);|
        \end{itemize}
\end{itemize}
\end{frame}

\begin{frame}[fragile,label=manualDeleteVector]{implementing vector: destroy}
\lstset{language=C++,style=small}
\begin{lstlisting}
template <class T> class MyVector {
    ...
private:
    T * array;
    int size, capacity;
};

template <class T>
void MyVector::pop_back(const T& other) {
    size--;
    array[size].~T();
}
\end{lstlisting}
\end{frame}
