\begin{frame}[label=localityIntro]{memory hierarchy assumptions}
\begin{itemize}
    \item \myemph{temporal locality} \\
        ``if a value is accessed now, it will be accessed again soon''
        \begin{itemize}
        \item caches should keep \myemph{recently accessed values}
        \end{itemize}
    \vspace{.5cm}
    \item \myemph{spatial locality} \\
        ``if a value is accessed now, adjacent values will be accessed soon''
        \begin{itemize}
        \item caches should \myemph{store adjacent values at the same time}
        \end{itemize}

    \vspace{1cm}
    \item natural properties of programs --- think about loops
\end{itemize}
\end{frame}

\begin{frame}[fragile,label=localityExamples]{locality examples}
\lstset{language=C,style=small}
\begin{lstlisting}
double computeMean(int length, double *values) {
    double total = 0.0;
    for (int i = 0; i < length; ++i) {
        total += values[i];
    }
    return total / length;
}
\end{lstlisting}
\begin{itemize}
    \item temporal locality: machine code of the loop
    \item spatial locality: machine code of most consecutive instructions
    \item temporal locality: {\tt total}, {\tt i}, {\tt length} accessed repeatedly
    \item spatial locality: {\tt values[i+1]} accessed after {\tt values[i]}
\end{itemize}
\end{frame}
