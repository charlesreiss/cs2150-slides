\begin{frame}[fragile,label=recurseToTail]{recursion to tail recursion}
\lstset{language=C++,style=smaller}
\begin{lstlisting}
int factorial_recursive(int x) {
    if (x <= 1)
        return 1;
    else
        return x * factorial_recursive(x-1);
}
\end{lstlisting}
\hrule
\begin{lstlisting}
int factorial_tail_recursive(int x, int y = 1) {
    if (x <= 1)
        return y;
    else
        return factorial_tail_recursive(x-1, x*y);
}
\end{lstlisting}
\end{frame}

\begin{frame}[fragile,label=noCall]{tail recursion: avoiding call}
\lstset{language=myasm,style=smaller,
        moredelim={**[is][\btHL<all:2>]{@2}{2@}}
}
\begin{lstlisting}
factorial_tail_recursive:
  cmp edi, 1
  jle .L4
.L2:
  imul esi, edi
  sub edi, 1
  @2jmp factorial_tail_recursive2@
  // same effect as:
  // call factorial_tail_recursive
  // ret
.L4:
  mov eax, esi
  ret
\end{lstlisting}
\end{frame}

\begin{frame}{tail recursion}
\begin{itemize}
    \item saves lots of stack space ($\Theta(x)$ space to $\Theta(1)$ space)
    \item easier for compilers to do
    \vspace{.5cm}
    \item ``tail'' requirement: must be last thing to do
    \item \ldots so it's okay to return directly to caller
\end{itemize}
\end{frame}

\begin{frame}[fragile,label=tailRecOnStack]{tail recursion: things on the stack}
\lstset{language=myasm,style=smaller,
        moredelim={**[is][\btHL<all:2>]{@2}{2@}}
}
\begin{lstlisting}
example_function:
    push rbx
    cmp rdi, 0
    je base_case
    ...
    ...
    @2pop rbx2@
    @2jmp example_function2@
base_case:
    pop rbx
    mov rax, ...
    ret
\end{lstlisting}
\end{frame}

\begin{frame}[fragile,label=tailToLoop]{tail recursion to loop}
\lstset{language=C++,style=smaller}
\begin{lstlisting}
int factorial_tail_recursive(int x, int y = 1) {
    if (x <= 1)
        return y;
    else
        return factorial_tail_recursive(x-1, x*y);
}
\end{lstlisting}
\hrule
\begin{lstlisting}
int factorial_loop(int x) {
    int y = 1;
    while (x > 1) {
        y *= x;
        x--;
    }
    return y;
}
\end{lstlisting}
\end{frame}
