\usetikzlibrary{graphs,graphs.standard,graphdrawing,quotes}
\usegdlibrary{force}


\begin{frame}{weighted graphs}
\begin{itemize}
\item some graphs have \textbf{weights} or \textbf{costs} associated with edges
\item example motivation:
    \begin{itemize}
    \item graph representing roads: weight = travel time
    \end{itemize}
\item \textbf{weight} or cost \textbf{of a path} = sum of weights of edges in path
\end{itemize}
\end{frame}

\begin{frame}{weighted graph example}
\begin{tikzpicture}
\tikzset{
    graphNode/.style={draw,thick},
    graphEdge/.style={draw,thick},
    >=Latex,
    my graph/.style={graphs/nodes={graphNode,label={[font=\small,fill=white,fill opacity=0.9]south:\tikzgraphnodetext}},graphs/edges={graphEdge},graphs/edge quotes={font=\small,auto,inner sep=0mm},
        graphs/typeset={},
    },
}
\begin{scope}[my graph,shift={(78,-38)},scale=4]
\graph[no placement] {
    Charlottesville[at={(-78.47,38.03)}],
    Culpeper[at={(-77.99,38.47)}],
    DC[at={(-77.01,38.90)}],
    Richmond[at={(-77.47,37.53)}],
    Fredericksburg[at={(-77.47,38.30)}],

    Charlottesville --["46"] Culpeper --["73"] DC[as={Washington, DC}],
    Charlottesville --["72"] Richmond --["58"] Fredericksburg --["53"] DC,
    Fredericksburg --["35"'] Culpeper,
};
\end{scope}
\end{tikzpicture}
\end{frame}
