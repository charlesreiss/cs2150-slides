\usetikzlibrary{graphs,graphs.standard,graphdrawing,quotes}
\usegdlibrary{force,trees}

\begin{frame}{breadth-first search}
\begin{itemize}
\item shortest path special case: weights = 1
\item algorithm is \myemph{breadth-first search}
\end{itemize}
\end{frame}

\begin{frame}{special case: breadth-first search on trees}
\begin{itemize}
\item can look at breadth-first search as generalization of pre-order traversal
\item difference: now we can have other paths from roots
\end{itemize}
\end{frame}

\begin{frame}[fragile,label=bfsIntuition]{breadth first search intuition}
\begin{itemize}
\item \myemph<2>{start with just source}
\item \myemph<3>{follow edges to first find vertices at distance 1}
\item \myemph<4>{then use those to find vertices at distance 2}, \myemph<5>{then distance 3}, \ldots
\end{itemize}
\begin{tikzpicture}
\tikzset{
    graphNode/.style={draw,circle,thick,font=\tt\fontsize{9}{10}\selectfont,minimum width=.6cm,inner sep=0.2mm},
    graphEdge/.style={draw,thick,alt=<7->{opacity=0.2}},
    >=Latex,
    my graph/.style={graphs/nodes={graphNode},graphs/edges={graphEdge},graphs/edge quotes={font=\tt\tiny,auto}},
    hilite/.style={alt=<#1>{red,very thick}},
    shortest path tree/.style={alt=<7->{opacity=1.0},alt=<7>{draw=red,very thick}},
}
\begin{scope}[my graph]
\graph[tree layout,level distance=10mm]{
    A[hilite=2] ->[shortest path tree] {[nodes={hilite=3}] B,C,D,E},
    C ->[shortest path tree] {[nodes={hilite=4}]G, H, I[> alt=<8>{draw=red,dotted}]},
    D -> {[nodes={hilite=4}]H, I[> alt=<8>{opacity=1.0,draw=red,dotted}], J[> shortest path tree]},
    E ->[bend right,hilite=6] {B, C},
    J ->[bend left,hilite=6] A,
    J ->[shortest path tree] {[nodes={hilite=5}] K, L},
    H ->[shortest path tree] M[hilite=5],
};
\begin{visibleenv}<6>
\node[right=1cm of E,align=left] {
    key idea: track \myemph{visited nodes} \\
    so we don't check them again \\
    (already found the shortest path) 
};
\end{visibleenv}
\begin{visibleenv}<7>
\node[right=1cm of E,align=left] {
    could have list of paths, one per node \\
    but more compact idea: \\
    store \myemph{one source edge per node} \\
    also called \textit{shortest path tree}
};
\end{visibleenv}
\begin{visibleenv}<8>
\node[right=1cm of E,align=left] {
    multiple possible answers!
};
\end{visibleenv}
\end{scope}
\end{tikzpicture}
\end{frame}

