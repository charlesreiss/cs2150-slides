\usetikzlibrary{graphs,graphs.standard,graphdrawing,quotes}
\usegdlibrary{force,circular,routing,layered,trees}


\begin{frame}{connectivity}
\tikzset{
    my graph node/.style={draw,circle,thick,font=\tt\fontsize{7}{8}\selectfont,minimum width=.3cm,inner sep=0.2mm,text=white},
    my graph edge/.style={draw,thick},
    my graph node hidden/.style={my graph node,draw=black!50},
    my graph edge hidden/.style={my graph edge,draw=black!50},
    >=Latex,
    my graph/.style={graphs/nodes={my graph node},graphs/edges={my graph edge},graphs/edge quotes={font=\tt\tiny,auto}},
    my graph hide/.style={graphs/nodes={my graph node hidden},graphs/edges={my graph edge hidden},graphs/edge quotes={font=\tt\tiny,auto}},
    graphs/show cycle/.style={nodes={draw=blue!70!black},edges={opacity=1.0,draw=blue!70!black}},
    graphs/show cycle B/.style={nodes={draw=orange!80!black},edges={opacity=1.0,draw=orange!80!black}},
    show cycle B/.style={draw=orange!80!black,graphs/nodes={draw=orange!80!black},graphs/edges={draw=orange!80!black}},
}
\begin{itemize}
\item \textbf{connected graph}: for all $x, y  \in V$, there exists a path from $x$ to $y$
    \begin{itemize}
    \item N.B: includes 0-length paths
    \end{itemize}
\end{itemize}
\begin{tikzpicture}
\matrix{
\begin{scope}[my graph]
\graph[spring layout] {
    [name=A] A -- B -- C -- D -- B -- D
};
\end{scope}
\node[draw,ultra thick,fit=(A A) (A D) (A C) (A B),label={north:a connected graph}] {};
\&
\begin{scope}[my graph]
\graph[spring layout] {
    [name=B] A -- B -- C,
    D, E -- F
};
\end{scope}
\node[draw,ultra thick,fit=(B A) (B D) (B C) (B B) (B E) (B F),label={north:a non-connected graph}] {};
\\
};
\end{tikzpicture}
\end{frame}

\begin{frame}{in a directed graph\ldots}
\begin{itemize}
\item \textbf{DAG} --- directed acyclic graph
    \begin{itemize}
    \item no cycles
    \end{itemize}
\item \textbf{strongly connected} --- path from every vertex to every other
    \begin{itemize}
    \item implies cycles (or digraph of 0 or 1 nodes)
    \end{itemize}
\item \textbf{weakly connected} --- would be connected as undirected graph
\end{itemize}
\end{frame}

\begin{frame}{strong/weak connected examples}
\begin{tikzpicture}
\tikzset{
    my graph node/.style={draw,circle,thick,font=\tt\fontsize{7}{8}\selectfont,minimum width=.3cm,inner sep=0.2mm},
    my graph edge/.style={draw,thick},
    my graph node hidden/.style={my graph node,draw=black!50},
    my graph edge hidden/.style={my graph edge,draw=black!50},
    >=Latex,
    my graph/.style={graphs/nodes={my graph node},graphs/edges={my graph edge},graphs/edge quotes={font=\tt\tiny,auto}},
    my graph hide/.style={graphs/nodes={my graph node hidden},graphs/edges={my graph edge hidden},graphs/edge quotes={font=\tt\tiny,auto}},
    graphs/show cycle/.style={nodes={draw=blue!70!black},edges={opacity=1.0,draw=blue!70!black}},
    graphs/show cycle B/.style={nodes={draw=orange!80!black},edges={opacity=1.0,draw=orange!80!black}},
    show cycle B/.style={draw=orange!80!black,graphs/nodes={draw=orange!80!black},graphs/edges={draw=orange!80!black}},
    mark graph/.style={draw,very thick},
}
\matrix (scA) {
\node[font=\small,align=center] {a strongly connected graph \\
drawn in two ways};
\\
\node {
\begin{tikzpicture}
\begin{scope}[my graph]
\graph[simple necklace layout] {
    [name=X]
    A -> B -> C -> D -> B -> E -> F ->  A,
    F -> G -> C,
};
\end{scope}
\node[fit=(X A) (X G) (X B) (X D) (X C) (X F),mark graph] {};
\end{tikzpicture}
};
\\
\node {
\begin{tikzpicture}
\begin{scope}[my graph]
\graph[spring layout] {
    [name=Y]
    A -> B -> C -> D -> B -> E -> F ->  A,
    F -> G -> C,
};
\end{scope}
\node[fit=(Y A) (Y G) (Y B) (Y D) (Y C) (Y F),mark graph] {};
\end{tikzpicture}
};
\\
};
\matrix[right=.5cm of scA] (scB) {
\node[font=\small,align=center] {another strongly\\connected graph};
\\
\node {
\begin{tikzpicture}
\begin{scope}[my graph]
\graph[tree layout] {
    [name=Z]
    // [simple necklace layout] { A -> B -> C -> D -> A },
    // [simple necklace layout] { E -> F -> G -> H -> I },
    C -> E,
    I -> C,
};
\end{scope}
\node[fit=(Z A) (Z G) (Z B) (Z D) (Z C) (Z F),mark graph] {};
\end{tikzpicture}
};
\\
};
\matrix[right=.5cm of scB] {
\node[font=\small,align=center] {a weakly connected graph };
\\
\node[mark graph] (weakGraph){
\begin{tikzpicture}
\begin{scope}[my graph]
\graph[spring layout,node distance=1.5cm] {
    [name=W] subgraph K_n [V={a,b,c,d},<->] -> subgraph C_n [V={e,f,g},<->],
};
\end{scope}
\begin{visibleenv}<2>
\node[fit=(W e) (W f) (W g),draw=red,dashed] {};
\node[fit=(W a) (W b) (W c) (W d),draw=red,dashed] {};
\end{visibleenv}
\end{tikzpicture}
};
\\
};
\begin{visibleenv}<2>
\node[below=.5cm of weakGraph,font=\small,red!70!black] {two \textit{strongly connected components}};
\end{visibleenv}
\end{tikzpicture}
\end{frame}
